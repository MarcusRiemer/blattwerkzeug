\section{Einleitung}
\label{sec:introduction}

In Deutschland scheint man Informatik als eine Hochschulangelegenheit zu betrachten. Anders ist es für mich kaum zu erklären, dass der Informatikunterricht an Schulen zwar durchaus vorgesehen ist, der Begriff ``Informatik'' dabei aber oftmals sehr weit interpretiert wird. Der Umgang mit Textverarbeitung, Tabellenkalkulation und Präsentationsmitteln wird heutzutage zwar an fast jeder Schule zumindest thematisiert, Inhalte mit tatsächlichem Bezug zur Informatik wie sie in einer späteren Ausbildung oder im Studium relevant relevant wären, sind dabei aber in der Minderheit.

Schaut man sich die für Informatikinhalte zur Verfügung stehenden Lehr- und Lernprogramme an (natürlich sollte Informatikunterricht auch am Rechner stattfinden!), verwundert diese stiefmütterliche Behandlung nur wenig: Die Menge an verfügbarer und gepflegter Software ist ausgesprochen überschaubar. Und natürlich ist es keinem Informatiklehrer zuzumuten, diese Lücke mit eigens geschriebener Software zu füllen.

Diese Arbeit ist der Versuch, eine Lehrsoftware für zwei wichtige Teilgebiete der Informatik anzubieten: Datenmodellierung und (Web-)Oberflächenentwicklung. Wenn man versucht das Ziel dieser Arbeit in einem Satz zu erklären, könnte sich dieser lesen wie folgt: Mit dem Blattwerkzeug lassen sich gestützt durch \textit{Drag \& Drop-Editoren} für beliebige \texttt{SQLite}-Datenbanken \textit{Abfragen formulieren} und \textit{Oberflächen entwickeln}.

\warning[Hinweis]{Der im Rahmen dieser Thesis fertiggestellte Stand des Prototypen kann unter \href{http://blattwerkzeug.de}{\texttt{blattwerkzeug.de}} ausprobiert werden. Um nur einen kurzen Blick auf das vorläufige Arbeitsergebnis zu werfen ist also keine lokale Installation notwendig. Da zum jetzigen Zeitpunkt allerdings noch keine Registrierung von Endanwendern implementiert worden ist, gestaltet sich der Zugriffsschutz ein wenig unbeholfen: Jedes Projekt dort verfügt über einen Testbenutzer namens \texttt{user} mit einem gleichlautenden Passwort. Dieser wird bei Speichervorgängen abgefragt, der lesende Zugriff ist auch ohne Zugang möglich.}

Aus praktischen Gründen sollte an dieser Stelle die Lernsoftware "`Scratch"' nicht unerwähnt bleiben: Der Arbeitstitel für die Thesis lautete "`Scratch für SQL und Webanwendungen"' und weckt bei Leuten, die mit Scratch schon vertraut sind, damit durchaus gewünschte, positive Assoziation. Wie Scratch versucht auch \idename{} viele typische und demotivierende Hürden wie Syntaxfehler, komplexe Entwicklungsumgebungen oder kryptische Fehlermeldungen konstruktiv auszuschließen. Kapitel \fullref{sec:related-work} beschreibt neben Scratch noch weitere, einflussreiche Inspirationsquellen.

Der wesentliche Anspruch an die im Rahmen dieser Arbeit zu erstellende Software ergibt sich also sowohl aus dem Titel der Arbeit als auch der Arbeitsweise von Scratch: Es geht vorrangig um die Vermittlung von praktischen Kenntnissen zur Abfrage, Manipulation und Visualisierung von komplexen Datenbeständen in Anlehnung an die Projektideen der Lehrpläne \cite{lehrplan-inf-sek-1} bzw. Fachanforderungen \cite{lehrplan-inf-sek-2} für Informatik des Landes Schleswig-Holstein\footnote{Die exakte Verortung im Curriculum oder auch die Konzeption von konkreten Schulstunden ist hingegen nicht Teil dieser Arbeit.}. Auch wenn sich aktuell eine zunehmende Pluralität von Paradigmen zur Datenspeicherung abzeichnet, welche das relationale Modell ergänzen oder in Frage stellen, behandelt diese Arbeit explizit die Vermittlung von \texttt{SQL}-Kenntnissen. Abbildung~\ref{fig:introduction-example-pokemon-go-query-editor} zeigt den Editor, mit dem in \idename{} Abfragen entwickelt werden können. Die Oberflächen werden ebenfalls in einem Drag \& Drop-Editor entwickelt, ein Beispiel dafür ist in Abbildung~\ref{fig:introduction-example-pokemon-go-page-editor} zu sehen.

\begin{figure}[h]
  \includegraphics[width=\textwidth]{images/screenshots/20161019/enduser-pokemongo-query-editor.png}
  \caption{Beispiel für eine mit \idename{} formulierte \texttt{SQL}-Abfrage.}
  \label{fig:introduction-example-pokemon-go-query-editor}
\end{figure}

\begin{figure}[p]
  \includegraphics[width=\textwidth]{images/screenshots/20161019/editor-page-full.png}
  \caption{Beispiel für eine mit \idename{} entwickelte Seite aus der Perspektive eines Entwicklers.}
  \label{fig:introduction-example-pokemon-go-page-editor}
\end{figure}

Die von den Schülerinnen und Schülern mit \idename{} erstellten Webauftritte werden auf den entwickelten Abfragen aufbauen und zumindest im Rahmen des in dieser Arbeit entwickelten Prototypen ein sehr einheitliches Erscheinungsbild haben: Die Ausgabe von Daten erfolgt in Tabellen, die Eingabe über Textfelder oder einfache Auswahlelemente (siehe Abbildung \ref{fig:introduction-example-pokemon-go-catch}). Optische Anpassungen sind nur in sehr eng gesteckten Grenzen möglich. Und genau aus diesem Grund ist im Untertitel von einer "`datenzentrierten Entwicklungsumgebung"' die Rede. In Bezug auf die Struktur der möglichen Datenmodelle ist die Flexibilität im Gegenzug nämlich ausgesprochen groß: Als Eingabe für ein \idename{}-Projekt dienen beliebige \texttt{SQLite}-Datenbank-Dateien. Der Anhang gibt dabei ein paar Hinweise, wie vielfältig die umsetzbaren Projekte daher letzten Endes sein können: Blogs, interaktive Geschichten, virtuelle Vitrinen für die eigene Sammelleidenschaft, ... Und natürlich ist diese Liste nicht abschließend zu verstehen!

\begin{figure}[p]
  \includegraphics[width=\textwidth]{images/screenshots/20161019/enduser-pokemongo-delete.png}
  \caption{Beispiel für eine mit \idename{} erstellte Seite aus Sicht eines Endbenutzers,  Abbildung~\ref{fig:introduction-example-pokemon-go-page-editor} zeigt die Entwickleransicht. Die Abfrage aus Abbildung~\ref{fig:introduction-example-pokemon-go-query-editor} wird für die tabellarische Auflistung verwendet.}
  \label{fig:introduction-example-pokemon-go-catch}
\end{figure}

Im Laufe dieser Arbeit stellt sich dabei wiederholt die Frage, wie man dieses nicht triviale Ziel einer "`selbstgebauten"' Internetseite durch geschickte Abstraktionen, Vereinfachungen und Hilfestellungen einerseits erreichbar macht, andererseits aber vor lauter praktisch motivierenden Ergebnissen nicht die eigentlichen Lernziele der Schülerinnen und Schüler aus den Augen verliert.

Diese Arbeit gliedert sich nach der Einleitung und der Vorstellung von Inspirationsquellen in zwei wesentliche Kapitel: \fullref{sec:requirements} analysiert auf Basis der vergleichbaren Arbeiten und unter Berücksichtigung eigener Ideen, wie eine solche "`praktische"' Lernsoftware für datengetriebene Anwendungen aussehen könnte. Da sich die dort gesammelten Funktionen nicht alle in dem für die Thesis zur Verfügung stehenden Zeitrahmen umsetzen lassen, ist das softwaretechnische Ergebnis dieser Thesis als ein "`minimum viable product"' zu verstehen. Die Umsetzung dieses Prototypen wird in Kapitel \fullref{sec:implementation-analysis} beschrieben. Kapitel \fullref{sec:conclusion} zieht dann Bilanz und wagt einen Ausblick: Welche Ziele wurden noch nicht erreicht, an welchen Stellen lohnt sich die Weiterentwicklung?

%%% Local Variables:
%%% mode: latex
%%% TeX-master: "thesis"
%%% End:
