\section{Der Name \idename{}}
\label{anh:the-name}

Das aktuell \idename{} genannte Programm wurde im Verlauf der Entwicklung mehrfach umbenannt: Erst lautete der Arbeitstitel "`Scratch for SQL"', dann "`esqulino"' und nun "`BlattWerkzeug"'. Letzterer Name entstand dabei als deutsche Variante eines möglichen englischen Titels, nämlich "`Pagetool"'.

Namen für die im Rahmen dieser Thesis erstellten Software müssen sich zunächst den folgenden Kriterien stellen:

\begin{enumerate}
\item Wenn man den Namen nur ausgesprochen hört, soll sofort eindeutig sein, welchen Begriff man in die Suchmaschine seiner Wahl eingeben soll.
\item Er sollte auf zumindest einen der beiden Anwendungszwecke hinweisen, also entweder auf Oberflächen mit \texttt{HTML} oder Datenbanken mit \texttt{SQL}. Optimalerweise verbindet der Name natürlich beide Einsatzzwecke.
\end{enumerate}

Leider erfüllen Weder "`esqulino"' noch "`BlattWerkzeug"' beide Kriterien so richtig zufriedenstellend.

Bei "`esqulino"' hängt man sehr stark an der \texttt{SQL}-Abkürzung, sofern man die nicht kennt fällt die Wiedererkennung schwer. Gleiches gilt für die vielen möglichen Schreibweisen wie "`SQLino"' oder "`esqlino"'. Und abgesehen davon ist ja auch die Sprechweise "`Sequel"' für \texttt{SQL} nicht ungebräuchlich.

Die Verbindung zu Internetseiten wird bei "`BlattWerkzeug"' in der Regel nur nach einer Erläuterung klar. Der gedankliche Weg von "`Page"' zu "`Seite"' und dann schließlich "`Blatt"' ist möglicherweise doch ein wenig weit hergeholt ...

Alles in allem ist der Name daher noch nicht notwendigerweise endgültig. Bessere Alternativen werden dankend entgegengenommen.

%%% Local Variables:
%%% mode: latex
%%% TeX-master: "thesis"
%%% End:
