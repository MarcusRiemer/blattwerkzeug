\begin{titlepage}

\section*{Abstrakt}

Konventionelle Entwicklungsumgebungen sind speziell auf die Bedürfnisse von professionellen Anwendern zugeschnittene Programme. Aufgrund der damit verbundenen Komplexität sind sie aus didaktischer Sicht nicht für die Einführung in die Programmierung geeignet. Diese Thesis beschreibt daher ein Konzept und die prototypische Implementierung einer Lehr-Entwicklungsumgebung names \idename{} für Datenbanken und Webseiten. Um syntaktische Fehler während der Programmierung systematisch auszuschließen, werden die Bestandteile der dafür benötigten Programmier- oder Textauszeichnungssprachen ähnlich wie in der Lehrsoftware "`Scratch"' grafisch durch Blockstrukturen repräsentiert. Diese Blöcke lassen sich über Drag \& Drop-Operationen miteinander kombinieren, die syntaktischen Strukturen von \texttt{SQL} und \texttt{HTML} sind für Lernende dabei stets sichtbar, müssen aber noch nicht verinnerlicht werden. So lassen sich auch ohne die manuelle Eingabe von Codezeilen eigene Webseiten programmieren, welche dann im Freundes- und Bekanntenkreis weitergegeben werden können. Für den Unterrichtseinsatz ist der aktuelle Entwicklungsstand von \idename{} allerdings noch nicht geeignet, er dient vornehmlich der Erprobung und Demonstration der erdachten Konzepte.

\section*{Abstract}

Conventional development environments are programs that are tailored to suit the needs of professionals. Due to their complexity they do not lend themselves well to introduce pupils to programming. This thesis therefore describes the concept and the prototypical implementation of an educational software named \idename{} (a loose translation of the english term ``Page Tool'') for database- and web-development. To eliminate the possibility of syntactical errors while programming, the elements of the programming- or markup-languages are represented by graphical blocks, similar to the approach taken by the software ``Scratch''. These blocks can be combined by using drag \& drop operations. The syntactical structures of \texttt{SQL} and \texttt{HTML} are not hidden from the user, but it is not mandatory to internalize them. This approach allows pupils to program and share their own websites, without the need to type lines of code. The current implementation of this software is not yet ready to be used in a classroom. It's main purpose is to demonstrate and field test the described concepts.

\end{titlepage}

%%% Local Variables:
%%% mode: latex
%%% TeX-master: "thesis"
%%% End:
