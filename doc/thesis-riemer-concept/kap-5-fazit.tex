\section{Fazit}
\label{sec:conclusion}

Ein Blick in den Anhang zeigt, dass der Prototyp dem eingangs erwähnten Ziel\footnote{"`Mit dem Blattwerkzeug lassen sich gestützt durch \textit{Drag \& Drop-Editoren} für beliebige \texttt{SQLite}-Datenbanken \textit{Abfragen formulieren} und \textit{Oberflächen entwickeln}"', siehe Kapitel \fullref{sec:introduction}} durchaus gerecht wird. Zu praktisch beliebigen SQLite-Datenbanken können Abfragen und Oberflächen entwickelt werden.

Etwas detallierter lässt sich der Grad des Erfolges einer prototypischen Implementierung vor allem am finalen Entwicklungsstand festmachen: Welcher prozentuale Anteil der angestrebten Funktionalität wurde erreicht? Daher muss sich der "`fertige Prototyp"' an den in Kapitel \fullref{sec:principles} formulierten Zielen messen lassen. Auch einige spätere Kapitel, insbesondere \fullref{sec:sql-subset}, lassen sich als ein recht detaillierter Anforderungskatalog verstehen.

Dieses vermeintlich objektive Kriterium berücksichtigt aber nur einen Teil der für \idename{} zu relevanten Ziele. Da auf dem prototypischen Stand weiter aufgebaut werden soll, ist eine weitere Frage von Bedeutung: Kann man auf diesem Stand die Weiterentwicklung fortführen?

\subsection{Erreichte Ziele}

Der aktuelle Stand der Implementierung ist vor allem als ein Durchstich zu verstehen: Auch wenn in fast jedem Teilbereich noch Funktionaliät fehlt, kann das Zusammenspiel dieser Systeme schon gut erprobt werden. Insbesondere bei der Verbindung der Datenbank mit der Oberfläche, sowohl lesend als auch schreibend, haben sich im Laufe der konkreten Implementierung noch einige unerwartete Stolpersteine aufgetan. Das zugrundeliegende Fundament, also die internen und textuellen Darstellungen von \texttt{SQL} und \texttt{HTML}, sind nun aber stabil und darüber hinaus auch mächtiger, als der mit dem Drag \& Drop-Editor editierbare Stand.

Das Grundprinzip "`\textbf{Semantik vor Syntax}"' wurde erreicht. Der Drag \& Drop-Editor Syntaxfehler kategorisch aus, die Fehlermeldungen für erkannte Fehlersituationen haben aber noch Verbesserungspotenzial. Aktuell werden fehlerhafte Eingaben einfach zugelassen und dann erst im Nachhinein als Fehler markiert. An dieser Stelle wäre das im Prinzip angesprochene "`kontinuirliche Feedback"' vermutlich am besten mit einer eingebauten, kontextsensitiven Hilfe unterstützen.

Ob die erstellbaren Seiten durch "`\textbf{praktisch vorzeigbare Ergebnisse motivieren}"', kann nur ein praktischer Test mit der Zielgruppe zeigen. Ein Fortschritt gegenüber der nicht sonderlich hohen Hürde "`besser sein als Texteingaben in einer \texttt{SQL}-\texttt{IDE}"' ist aber durchaus zu erwarten.

Die "`\textbf{einfache Inbetriebnahme}"' ist eine Frage der Perspektive. Aus Sicht eines Schülers ist der einfache Aufruf einer \texttt{URL} in der Tat einfach, bisher hat \idename{} auch gut unter allen ad-hoc probierten Kombinationen aus Betriebssystemen (Windows, Mac\-OS, Linux) und Browsern (Firefox, Chrome, Edge) gut funktioniert.

Das technische Fundament aus Ruby mit Sinatra und Typescript mit Angular 2 hat sich ebenso bewährt, wie die Entscheidung den größten Teil der Logik im Client zu belassen. Der Betrieb von \idename{} fühlt sich nach der initialen Ladezeit sehr flüssig an. Die Möglichkeit der Entwicklung von Unit- und End-to-End-Tests fügt sich gut in den Entwicklungszyklus ein.

\subsection{Nicht erreichte Ziele}

Das Ziel einer "`\textbf{schrittweise komplexeren Benutzeroberfläche}"' wurde zunächst hinten angestellt, da es mit einem ausgearbeiteten Konzept einhergehen sollte. Kapitel \fullref{sec:sql-subset-ranks} untersucht zwar die möglichen Einschränkungen für \texttt{SQL}, geht aber nicht auf \texttt{HTML} oder die möglichen Auswirkungen auf das Zusammenspiel beider Systeme ein. Mit dem aktuell noch überschaubaren Funktionsumfang ist der akute Bedarf nach diesem spezifischen Grundprinzip allerdings auch noch nicht gegeben.

Auch das Ziel der "`\textbf{Fortführung der entwickelten Projekte}"' mit externen Programmen wurde für den Prototypen hinten angestellt. Und zumindest die Unterstützung der Quelltext-Editoren sollte noch implementiert werden, bevor der Bedarf für einen Export überhaupt abgeschätzt werden kann.

Der Umfang der tatsächlich implementierten \texttt{SQL}-Funktionen ist noch nicht abgeschlossen. Zum jetzigen Zeitpunkt fehlen neben der Unterstützung der \texttt{AS}-Direktive im Editor (der Syntaxbaum unterstützt die Benennung schon) vor allem Funktionen und Gruppierungen.

\subsection{Einsatz im Unterricht}

\subsection{Weiterentwicklung}

Eine wesentliche Frage bei der Weiterentwicklung von \idename{} ist die Frage nach dem Zeitpunkt, zu dem man die eigentliche Zielgruppe (sowie deren Lehrer) in die Entwicklung mit einbezieht. Oder anders ausgesdrückt: Es stellt sich die Frage nach dem minimalen Satz an Funktionen, mit denen man sinnvoll Feedback einholen kann.

\begin{description}
\item[SQL: Unterstützung von Funktionen] \hfill\\

\item[SQL: Unterstützung von \texttt{GROUP BY}] \hfill\\

\item[SQL: Schemaeditor] \hfill\\
  
\end{description}




%%% Local Variables:
%%% mode: latex
%%% TeX-master: "thesis"
%%% End:
