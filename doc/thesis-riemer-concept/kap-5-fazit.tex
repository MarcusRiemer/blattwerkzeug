\section{Fazit}
\label{sec:conclusion}

Ein Blick auf die im Anhang vorgestellten Projekte (\fullref{sec:project-examples}) zeigt, dass der Prototyp dem eingangs erwähnten Ziel\footnote{"`Mit dem Blattwerkzeug lassen sich gestützt durch \textit{Drag \& Drop-Editoren} für beliebige \texttt{SQLite}-Datenbanken \textit{Abfragen formulieren} und \textit{Oberflächen entwickeln}"', siehe Kapitel \fullref{sec:introduction}} durchaus gerecht wird: Zu praktisch beliebigen SQLite-Datenbanken können Abfragen und Oberflächen entwickelt werden.

Etwas detaillierter lässt sich der Grad des Erfolges einer Software vor allem am finalen Entwicklungsstand festmachen: Welcher prozentuale Anteil der angestrebten Funktionalität wurde erreicht? Daher muss sich der "`fertige Prototyp"' an den in Kapitel \fullref{sec:principles} formulierten Zielen messen lassen. Auch einige spätere Kapitel, insbesondere \fullref{sec:sql-subset}, lassen sich als ein recht umfassender Anforderungskatalog verstehen. Die formulierte Roadmap (siehe \fullref{sec:implementation-roadmap}) wurde bis auf den letzten Punkt "`Qulitätssicherung"' eingehalten. Eine Ausweitung der in Kapitel \fullref{sec:implementation-tests} beschriebenen Testmethodiken steht also ebenfalls noch aus.

Dieses vermeintlich objektive Kriterium der prozentualen Vollständigkeit berücksichtigt aber nur einen Teil der für Prototypen wie  \idename{} relevanten Ziele. Da in diesem Fall auf dem prototypischen Stand weiter aufgebaut werden soll, ist eine weitere Frage von Bedeutung: Kann man auf diesem Stand die Weiterentwicklung fortführen?

\subsection{Erreichte Ziele}

Der aktuelle Stand der Implementierung ist vor allem als ein Durchstich zu verstehen: Auch wenn es in fast jedem Teilbereich noch vereinzelt an Funktionalitäten fehlt, kann das Zusammenspiel dieser Systeme schon gut erprobt werden. Insbesondere bei der Verbindung der Datenbank mit der Oberfläche, sowohl lesend als auch schreibend, haben sich im Laufe der konkreten Implementierung noch einige unerwartete Stolpersteine aufgetan (\fullref{sec:unexpected-problems}). Das zugrundeliegende Fundament, also die internen und textuellen Darstellungen von \texttt{SQL} und \texttt{HTML}, sind nun aber stabil und darüber hinaus auch mächtiger, als der mit dem Drag \& Drop-Editor editierbare Stand.

Das Grundprinzip "`\textbf{Semantik vor Syntax}"' wurde erreicht, der Drag \& Drop-Editor schließt Syntaxfehler kategorisch aus. Sofern diese im kompilierten Quelltext doch auftreten, wäre das eindeutig ein Fehler in der Codegenerierung, nicht jedoch des Entwicklers. Bei Fehlermeldungen für erkannte Fehlersituationen gibt es jedoch noch Verbesserungspotenzial: Aktuell werden fehlerhafte Eingaben einfach zugelassen und dann erst im Nachhinein als Fehler markiert. An dieser Stelle wäre das im Prinzip angesprochene "`kontinuierliche Feedback"' vermutlich am besten mit einer eingebauten, kontextsensitiven Hilfe unterstützen. Diese könnte auf Fehler mit einer kurzen Erläuterung reagieren und dabei demonstrieren, wie die richtige Vorgehensweise wäre.

Ob die erstellbaren Seiten durch "`\textbf{praktisch vorzeigbare Ergebnisse motivieren}"', kann nur ein Test mit Probanden der Zielgruppe zeigen (siehe \fullref{sec:target-audience}). Ein Fortschritt gegenüber der nicht sonderlich hohen Hürde "`besser sein als Texteingaben in einer \texttt{SQL}-\texttt{IDE}"' ist aber durchaus zu erwarten.

Die "`\textbf{einfache Inbetriebnahme}"' ist vor allem eine Frage der Perspektive. Aus Sicht eines Schülers ist der Aufruf einer \texttt{URL} in der Tat einfach. Bisher hat \idename{} unter allen ad-hoc probierten Kombinationen aus Betriebssystemen (Windows, Mac\-OS, Linux) und Browsern (Firefox, Chrome, Edge) gut funktioniert. Für Lehrkräfte wird aktuell eine virtuelle Maschine bereitgestellt. Der Umgang mit dieser ist aufgrund des fehlenden Webinterfaces allerdings noch relativ unbequem.

Eine Notwendigkeit des Wechsels auf eine "`normale"' Desktopanwendung hat sich zu keinem Zeitpunkt ergeben. Das technische Fundament aus Ruby mit Sinatra und Typescript mit Angular 2 hat sich ebenso bewährt, wie die Entscheidung, den größten Teil der Logik im Client zu belassen. Der Betrieb von \idename{} ist nach der initialen Ladezeit fast vollständig frei von Verzögerungen. Die Möglichkeit der Entwicklung von Unit- und End-to-End-Tests fügt sich gut in den Entwicklungszyklus ein.

\subsection{Nicht erreichte Ziele}

Das Ziel einer "`\textbf{schrittweise komplexeren Benutzeroberfläche}"' wurde zunächst hinten angestellt, da es mit einem ausgearbeiteten Konzept einhergehen sollte. Kapitel \fullref{sec:sql-subset-ranks} untersucht zwar die möglichen Einschränkungen für \texttt{SQL}, geht aber nicht auf \texttt{HTML} oder die möglichen Auswirkungen auf das Zusammenspiel beider Systeme ein. Mit dem aktuell noch überschaubaren Funktionsumfang ist der akute Bedarf nach diesem spezifischen Grundprinzip allerdings auch noch nicht gegeben.

Zudem wurde das Ziel der "`\textbf{Fortführung der entwickelten Projekte}"' mit externen Programmen im Rahmen des Prototypen hinten angestellt. Zumindest die Unterstützung der Quelltext-Editoren sollte noch implementiert werden, bevor der Bedarf für einen Export überhaupt abgeschätzt werden kann.

Der Umfang der tatsächlich implementierten \texttt{SQL}-Funktionen ist noch nicht abgeschlossen. Zum jetzigen Zeitpunkt fehlen neben der Unterstützung der \texttt{AS}-Direktive im Editor (der Syntaxbaum unterstützt die Benennung bereits) vor allem Funktionen und Gruppierungen. Dieser Umstand schränkt die umsetzbaren Projekte empfindlich ein: Stünden diese Funktionen schon zur Verfügung, könnten zum Beispiel auch Webseiten für Sportvereine, inklusive der dynamischen Erzeugung von Tabellen, mit \idename{} umgesetzt werden.

Der Einsatz des Prototypen im Unterricht ist aktuell vor allem aufgrund der rudimentären Benutzerverwaltung nur schwer vorstellbar. Noch ist keine Registrierung von Benutzern implementiert, außerdem können Projekte nicht über die Webseite kopiert werden. Die Implementierung einer Benutzerverwaltung ist allerdings eine vor allem handwerkliche Aufgabe und wurde daher im Rahmen dieser Thesis nicht vorangetrieben.

Aber auch der Betrieb für Lehrkräfte gestaltet sich deutlich komplizierter als erwünscht: Die Notwendigkeit einer eigenen (Sub-)Domain ist keine triviale Hürde. Die Annahme "`man wird doch wohl mal auf dem Nameserver der Schuldomain einen Eintrag für \idename{} anlegen können"' hat sich zwar nicht als haltlos erwiesen, behindert die initiale Inbetriebnahme aber merklich.

\subsection{Weiterentwicklung}

Eine offene Frage bei der Weiterentwicklung von \idename{} ist die Wahl des Zeitpunkts, zu welchem die eigentliche Zielgruppe (sowie deren Lehrkräfte) in die Entwicklung mit einbezogen werden. Oder anders ausgedrückt: Es stellt sich die Frage nach dem minimalen Satz an implementierten Funktionen, mit denen man sinnvoll Feedback bei Lehrern und Schülern einholen kann.

Die folgende Aufzählung von offenen Baustellen gibt vor allem Anhaltspunkte, an welchen Stellen die prototypische Implementierung unvollständig ist. Diese Punkte sind dabei notwendig für einen tatsächlichen Einsatz im Unterricht, aber nicht zwingend hinreichend. Insbesondere die Erfüllung von weichen Kriterien, zum Beispiel eine umfassende Qualitätssicherung oder bessere Dokumentation für Schüler \& Lehrkräfte, werden nicht aufgeführt.

\begin{description}[noitemsep]
\item[Allgemein: Textuelle Editoren] \hfill\\
  Momentan begrenzen die Drag \& Drop-Editoren den Leistungsumfang ganz erheblich: Das interne Datenmodell ist hingegen darauf ausgelegt, auch textuelle Darstellungen "`blind"' entgegenzunehmen. Um Frust aufgrund von (noch?) nicht unterstützten Ideen zu vermeiden, sollten daher Texteditoren als "`Notausgang"' implementiert werden. Im Falle von \texttt{HTML} ist das über das spezielle Code-Bedienelement schon in Ansätzen möglich, für \texttt{SQL} hingegen noch überhaupt nicht.
  
\item[SQL: Unterstützung für Funktionen und \texttt{GROUP BY}] \hfill\\
  Ohne Aggregation von Daten lassen sich viele alltägliche Fragestellungen nicht beantworten. Glücklicherweise ist die eigentliche Implementierung nicht besonders aufwändig: Der Syntaxbaum ist darauf vorbereitet, für die Oberfläche ergeben sich keine bisher nicht schon gesehenen Anforderungen.
  
\item[SQL: Schemaeditor] \hfill\\
  Die Umsetzung eigener Ideen ist mit dem gegenwärtigen Stand faktisch nicht durchführbar, da es keine Möglichkeit gibt, das Datenbankschema über die Weboberfläche zu editieren. Eine Minimallösung wäre der einfache Upload von SQLite-Dateien, dann erfordert aber jede Anpassung des Schemas den Wechsel in ein externes Werkzeug.

  Die optimale Lösung wäre ein in \idename{} integrierter Schema-Editor. Mit diesem sollte man mindestens Tabellen und Spalten anlegen, umbenennen sowie löschen können. Aufwändig, aber sehr interessant wäre noch die Definition von Beziehungen.

\item[Seiten: Drag \& Drop für interpolierte Ausdrücke] \hfill\\
  Gegenwärtig müssen Liquid-Objekte unmittelbar in geschweiften Klammern im Quelltext notiert werden, es findet nicht einmal Syntax-Highlighting statt. Dieser Stand bricht daher nicht nur mit dem Drag \& Drop-Paradigma, er verwirrt auch mit schwer nachvollziehbaren Meldungen bei Syntaxfehlern. Folglich sollte in einer zukünftigen Version auch für diesen Bereich Drag \& Drop-Unterstützung durch \idename{} erfolgen.
  
\item[Seiten: Drag \& Drop für Kontrollstrukturen] \hfill\\
  Bisher ist der Entwickler darauf angewiesen, dass für Wiederholungen spezielle Bedienelemente wie die Datentabelle zur Verfügung gestellt werden, Verzweigungen lassen sich momentan außerhalb des \texttt{HTML}-Text-Elementes überhaupt nicht nutzen. Um diesem Umstand zu begegnen, sollten die Liquid-Tags \texttt{for} und \texttt{if} wie die vorhandenen Bedienelemente im Drag \& Drop-Editor repräsentiert werden können.  
\end{description}

Und abseits von den Details der Editoren für \texttt{HTML} und \texttt{SQL} ist auch noch eine Grundsatzentscheidung zu treffen: Die Form, in der \idename{} Lehrkräften zugänglich gemacht werden soll. Das Feedback aus informellen Gesprächen gibt zumindest Anlass, die eingangs formulierte Prämisse "`jeder Lehrer stellt halt einen Server für seine Klassen bereit"' erneut zu hinterfragen. Ein dazu häufig formulierter Kritikpunkt lässt sich mit "`Immerhin wird Scratch auch einfach so im Web angeboten"' zusammenfassen. Dieser Ansatz wäre natürlich auch denkbar, eine solche Entscheidung hat aber enorme Tragweite und wirft viele Fragen auf: Kann die Anmeldung über einen Dienst der Schulen erfolgen? Was für Auswirkugen hat das auf interessierte Schülerinnen ohne eine Schule im Hintergrund? Sollen die Schüler unterschiedlicher Schulen miteinander interagieren können? Wer haftet bei Verstößen gegen das Urheberrecht? Wer bezahlt den Betrieb der notwendigen Server? All diese (und vermutlich noch weitere) Fragen werden im Falle einer Umstellung zu klären sein.




%%% Local Variables:
%%% mode: latex
%%% TeX-master: "thesis"
%%% End:
