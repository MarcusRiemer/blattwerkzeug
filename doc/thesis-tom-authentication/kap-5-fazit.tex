\section{Fazit}
\label{sec:conclusion}
Das Fazit befasst sich kurz mit den erreichten Zielen und den möglichen Erweiterungen die aus dieser Thesis resultieren. 

\subsection{Erreichte Ziele}
Im Rückblick auf die Anforderungen dieser Arbeit wurden alle Ziele erreicht. Dabei ist eine Anmeldung mittels Passwort und \gls{oAuth2} realisiert worden. Die Anmeldung mittels \gls{oAuth2} ist über die Provider Google und GitHub möglich. Für eine Anmeldung mittels Passwort wurde eine zusätzliche Möglichkeit zur Registrierung hinzugefügt.

Ebenfalls realisiert werden konnte die serverseitige Überprüfung von Benutzerrechten. Für das Erstellen, Löschen und Modifizieren von Projekten oder News wurde diese Überprüfung bereits implementiert. Desweiteren wurde der bedingte Zugriff auf clientseitige Routen und Bedienelemente ebenfalls realisiert.

Ebenso eingebaut werden konnten die Einstellungen im Bereich Sicherheit und Login. Mittlerweile ist es möglich sein bereits erstelltes Konto mit weiteren Konten zu verknüpfen. Zusätzlich können die bereits verknüpften Konten verwaltet werden. Der Wechsel eines Passwortes und eines Benutzersnamens ist inzwischen ebenfalls durchführbar.

\subsection{Ausblick}
Im Hinblick auf die Zukunft haben sich während der Ausarbeitung der Thesis bereits mehrere Ideen zur Weiterentwicklung ergeben. 

Zu diesem Zeitpunkt ist die Autorisierung von Bedienelementen inifizient, we{\ss}halb für die Zukunft bereits eine Weiterentwicklung der zuversendenen \gls{HTTP}-Anfragen geplant ist. Hierbei werden \gls{HTTP}-Anfragen innerhalb von einer Sekunde zusammen gefasst und als eine Anfrage versendet. Folglich würde die Serverauslastung durch übermä{\ss}ige Anfragen reduziert werden. Ein Teil dieser Funktionalität wurde bereits implementiert. Dafür wurde, wie in Sektion \ref{sec:server-may-perform} beschrieben, statt einem einzelnen Wert ein Array verarbeitet.

Desweiteren ist das Hinzufügen weiterer statischer Rollen geplant, die spezifisch für beispielsweise Lehrer oder Schüler angewendet werden. Da Blattwerkzeug weitestgehend an Schulen auftreten soll, kann mit der Möglichkeit zur Autorisierung weiterer Inhalt für unterschiedliche Benutzergruppen erstellt werden.

Eine weitere geplante Weiterentwicklung ist das Anlegen einer Profilseite für jeden registrierten Benutzer. Die Einstellungsmöglichkeiten eines Profils könnten hierbei in dem bereits erstellten Einstellungs-Modul ergänzt werden. 

Darüber hinaus bietet die Implementierung von Authentifizierung und Autorisierung die Basis zur Entwicklung weiterer Systeme. Ein Beispiel für eine derzeit nicht geplante Erweiterung wäre ein Nachrichten-System zur Kommunikation zwischen Benutzern.


