\section{Abstract}
\label{sec:abstract}

Augmenting \enquote{normal} programming languages to be syntax free and better accessible for people learning to program.


\section{Introduction}
\label{sec:introduction}

Augmenting \enquote{normal} programming languages to be syntax free and better accessible for people learning to program.

\section{Syntaxfree Programming}
\label{sec:syntaxfree-programming}

Common high level programming languages like Ruby, JavaScript, C\#, Haskell, Lisp~... all share a characteristic that usually goes without saying: Their source code is stored as plain text files. This is an obviously proven model that has many benefits regarding the ease to read or edit these files: Any text editor is sufficient. For people new to programming this however imposes a serious drawback: Text files are not very discoverable. Environments that teach the absolute basics semantics of programming (sequences, alternatives and jumps) therefore often resort to radically different program representations.

One such program is Lightbot \cite{yaroslavski_lightbot}. It requires the user to guide a little robot through various worlds initially using seven commands: \texttt{forward}, \texttt{turnLeft}, \texttt{turnRight}, \texttt{jump}, \texttt{light}, \texttt{proc1} and \texttt{proc2}. The fundamental idea of this approach is very similar to \enquote{Turtle Programs} as described by \cite{papert_mindstorms_1982}, it however adds an environment that requires the user to solve puzzles.

And in contrast to a classic turtle program, a lightbot program is not a textfile that requires the user to enter these commands by hand. Instead the commands are little, graphical blocks that are inserted via click or drag \& drop (see \autoref{fig:lightbot}). No matter how hard a programmer tries to create a syntactically incorrect program: It's just not possible. On top of that every action is directly represented as a UI element that can be interacted with.

\begin{figure}
  \includegraphics[width=\textwidth]{images/lightbot-example}
  \caption{Lightbot syntaxfree programming environment}
  \label{fig:lightbot}
\end{figure}

The same concept is described as \enquote{Mini Languages} by \cite{brusilovsky_mini-languages_1997} which defines properties that such languages should have:

\begin{itemize}
\item TODO
\end{itemize}

The probably most common implementation of these ideas of syntaxfree programming and mini languages is Scratch which originated in 2004 \cite{maloney_scratch_2004} and is still under active development. Instead of being restricted to turtle like graphics or puzzle solving, it provides a digital canvas on which students can program interactions between actors, backgrounds and user input. From a programmers point of view, Scratch defines its own, concurrent programming language and standard library which is focused on graphics-development related tasks.

\subsection{Transferring Skills to Text-Based Programming}

While this approach is proven to be an excellent way to spark interest in programming, it is not entirely clear how big the effect for .

\section{Overview: From Grammar to IDE}

This chapter is an outline to the IDE generation process as a whole and introduces most of the relevant concepts on a level that is beneficial to explain the connections between the various concepts, data structures and transformations. The details of these steps are explained later in the document, but this section will contain forward-references so that you can jump into the details if you prefer to do so right away.

\subsection{Differentiation from Traditional Compiler Construction}
\label{sec:diff-traditional-compiler}

The term \enquote{Grammar} is of fundamental importance for this whole article and the whole field of compiler construction. In mathematical terms a grammar $G$ is usually described as a 4-tuple $G = (V, T, P, S)$ where $V$ is a set of variables, $T$ is a set of terminal symbols, $P$ are the so called \textit{productions} and $S$ is the starting symbol\cite[Chapter 5]{hopcroft_formal_languages}. We can than use a grammar $G$ to check any text input whether it is valid in the context of the grammar.

Apart from using a grammar to classify whether a certain text-input is well formed, the \textit{parser tree} is an important by-product of the parsing process. In compiler construction this tree is usually known as the first \textit{intermediate representation} or an \textit{Abstract Syntax Tree} (AST)\cite[Chapter 6]{dragon_book}.

This intermediate representation may then be transformed to e.g. optimize for speed, size or any other goal. But all of these transformations usually transform on AST into another AST. The result may not be compatible in the sense of being the exact same datatype, but it will still be a tree. During a process called \textit{code generation}, the AST is then transformed into some machine-readable format like binary \texttt{x86} machine code\cite[Chapter 8]{dragon_book}.

Image: \texttt{Source Code (Text) -> AST -> Machine Code}

And while the topic of this dissertation also is a compiler, it has a quite different set of datatypes for input and output. A syntax-free program as described in \fullref{sec:syntaxfree-programming} is not read in is text-form by the user, but instead presented through a visualisation program. This allows the designer of a syntax-free language to optimize the data representation for a compiler, instead of a human. Commonly these programs are therefore stored in data exchange formats like \texttt{XML} or \texttt{JSON}. Which in turn means, that the whole parsing process (which is covered by more than 300 pages in \cite{dragon_book} is almost irrelevant for this article.

The result of the compilation process is also different: As the aim of this work is to augment \textit{existing} programming languages (see \fullref{sec:introduction}), the result of the code generation is not machine code, but source code. Which again means that the remaining 700 pages of \cite{dragon_book} are not directly of interest: The book provides the framing of high level concepts and terms for this work, but the machine-oriented details are not relevant in this specific context.

Image: \texttt{AST -> Source Code}

So in the context of this dissertation, the term compiler doesn't refer to a traditional compiler like \texttt{gcc}, \texttt{javac} or \texttt{ghc}: Instead the term is used in the most general sense as outlined on the first page of \cite{dragon_book}: \enquote{Simply stated, a compiler is a program that can read a program in one language, the source language, and translate it into an equivalent program in another language, the target language}.

\subsection{Comparision to Programming Language Engineering Tools}

Compiler construction is a mature field and there are various tools ease programming language engineering. The typical low level tools like \texttt{yacc}, \texttt{flex}, \texttt{bison} or higher level abstractions like \texttt{llvm} are not relevant for this work because they solve problems of parsing and code generation as outlined in \fullref{sec:diff-traditional-compiler}. But with the increasing popularity to provide \textit{Domain Specific Languages} instead of general purpose programming or configuration languages a set of much higher level tools has surfaced \cite{mernik_dsl_2005}.

One of the tools that was a direct inspiration is the Xtext Framework for Eclipse \cite{efftinge_xtext}. The fundamental idea of Xtext is to provide a grammar file as a basis for the programming language that is to be designed. This grammar can then serve as a basis that enables a language designer to provide many of the convenience features that developers expect from their IDE. Aspects like syntax highlighting, error markers, automatic indentation, navigation in the source code by jumping to definitions and code completion can all be implemented for Eclipse with reduced effort compared to an implementation from scratch.

\subsection{Syntaxtree}

All operations of the generated syntaxfree IDEs have to use the same basic data structure for the AST. It must not matter whether the augmented language has very significant whitespace like Python or Haskell, uses (very (many) (pairs) of brackets) like Lisp or is not actually a programming language but a mathematical formula, ... The goal is to define a tree that can represent any kind of text. Therefore the concept of the described syntaxtree is based on \texttt{XML}\cite{xml_spec}.

A \textit{node} in the AST is a recursive data structure that can store two things:

\begin{itemize}
\item Named, atomic \textit{properties} in the form of \texttt{number}, \texttt{string} or \texttt{boolean} values. The analogous \texttt{XML}-concept here would be the \textit{attribute}, which is denoted as a \texttt{key='value'}.
\item Other \textit{nodes} as child nodes. The analogous \texttt{XML}-concept here are nested elements, denoted as \texttt{<parent><child></child></parent>}.
\end{itemize}

TODO: Name, language and Named children.

The used syntax to store these trees is however simply \texttt{JSON}

\subsubsection{Benefits compared to text representation}

\subsection{Validation}

Now that we have seen how to define ASTs, the natural follow up question \enquote{Is this particular AST a valid program?} In traditional compiler construction the first step is to define a grammar in notations like the (extended) Backus-Naur-Form \cite{knuth_backus_1964}. Not re-inventing the wheel is an important aspect of computer science, so re-using a common notation would seem like a valid approach. But in the case of this dissertation, the Backus-Naur-Form is not directly applicable: Its result is an AST, but as we already haven an AST to begin with there is no reason to be concerned with the \texttt{String -> AST} transformation. As the general structure of the AST is already loosely modeled after \texttt{XML} it seems natural to dive into this domain for validation.

\subsubsection{XML Schema / Relax NG}

As one of the de-facto standard formats for data exchange the question \enquote{Does this particular \texttt{XML}-document contain the data in a structure that is expected?} arises every time a system receives an \texttt{XML} document. Or to put the question in more technical terms: \enquote{Is this document \textit{well-formed} with respect to a certain \textit{schema}?}

An \texttt{XML} Schema (\texttt{XSD}) \cite{xml_schema} can be seen as a datatype definition or as a grammar, from an abstract viewpoint these terms are interchangeable \cite[Chapter 5.3.4 \enquote{XML and Document Type Definitions}]{hopcroft_formal_languages}. But as an \texttt{XML}-schema is used to validate inputs for programming languages, it provides a way to define that typical datatypes like \texttt{string}, \texttt{integer}, \texttt{float}, \texttt{boolean}, ... are expected in the document, e.g. as attributes. It is also possible to directly generate e.g. a Java \texttt{class} from an \texttt{XML} schema.

The details of \texttt{XML} schema are quite complex, but from a birds-eye perspective the rules for child elements can be defined in three different ways:

\begin{itemize}
\item \texttt{sequence} expects child nodes to be present in an exact order. The expected cardinality of each item may be specified for each child.
\item \texttt{choice} is a way to describe mutually exclusive children: A child node may be either one of the provided items, but not none or multiple ones.
\item \texttt{all} treats the child node as an unordered multiset: The expected cardinality of each item may be specified, but the order does not matter.
\end{itemize}

An \texttt{XML} schema is itself an \texttt{XML} document and therefore trivially parseable but sometimes a little convoluted to read. The alternative RELAX NG syntax defines a compact syntax (or for that matter: a domain specific language) that is semantically equivalent but easier to read.

\subsubsection{Semantic Analysis}

Semantic analysis of an AST strongly depends on the language that is being represented. Rules for e.g. typical errors such as undeclared variables differ from language to language. One way of defining these relationships is via so called \textit{attribute grammars} \cite{knuth_semantics_1968}. Knuth describes the goal for such a formalism as follows: \enquote{All of the known methods for defining the meaning of computer programs were
based on rather intricate algorithms having roughly the same degree of complexity as
compilers, or worse. This was in stark contrast to Chomsky’s simple and elegant method
of syntax definition via context-free grammars.} \cite{knuth_genesis_1990}.

On top of that semantic analysis may require information that is not part of the AST itself: In \texttt{SQL} for example the same query may be valid in database \textit{A} but invalid in database \textit{B}. This depends on the names of the tables and columns and their respective datatypes. Therefore semantic analysis in the context of this dissertation is done through ordinary code, TypeScript to be exact.



\subsection{Grammar}

\subsubsection{\texttt{node}}

\subsubsection{\texttt{property}}

\subsubsection{children: \texttt{sequence}}

\subsubsection{children: \texttt{allowed}}

\subsubsection{\texttt{typedef}}

\subsection{Visual Grammar}

\subsubsection{\texttt{interpolate}}

\subsubsection{\texttt{each}}

\subsubsection{\texttt{container}}

\subsection{BlockLanguage}

\subsubsection{Node representation}

\subsubsection{Available Blocks (Sidebars)}

\subsubsection{Language Levels}

As outlined by \cite{klaeren_macht_2007}.

\subsection{Generation: Target Source Code}

As outlined in \fullref{sec:diff-traditional-compiler} the term \enquote{Code Generation} does not refer to traditional generation of machine code. Instead the generated code is assumed to be human readable text in any higher level language. The provided primitives for code generation are therefore concerned with concepts like indentation, pair-wise insertion of brackets and insertion of delimiters like commas or semicolons. \fullref{sec:code-generation-detail} explains the details of the \texttt{AST -> String} in more technical detail.

The source code that is generated from the AST must follow the established standards of the target language as good as possible. This is an explicit goal, even if it wouldn't matter for the current language runtime. The (admittedly a little tongue-in-cheek) \enquote{Law of Leaky Abstractions} \cite{spolsky_law_2002} predicts, that any abstraction at some point may break. Or in the context of this dissertation: It must not be assumed that the generated target source code is always flawless, therefore glossing over it (or worse: debugging it) must be as frictionless as possible. And on top of that a \enquote{nice} code generation could provide students with a viable way to take their program out of the syntaxfree IDE.

\subsubsection{Target Language Formatters}

As the goal of this project is the augmentation of existing programming languages, it would be possible to tap into the vast ecosystem of modern source code formatters that are available for almost any language. One could for example use a tool HTML Tidy\footnote{\url{https://www.html-tidy.org/}} to process the following snippet that is perfectly fine from the perspective of a browser:

\begin{minted}{html}
<html><head><title>My Page</title></head><body><h1>Hi</h1></body></html>
\end{minted}

After using HTML Tidy, the source code could be formatted as follows:

\begin{minted}{html}
<html>
  <head>
    <title>My Page</title>
  </head>
  <body>
    <h1>Hi</h1>
  </body>
</html>
\end{minted}

Technically speaking this approach is a \texttt{String -> String} transformation that would have to be implemented once for every target language: HTML could use HTML tidy, JavaScript could use Prettier\footnote{\url{https://prettier.io/}}, Ruby could use Rubocop\footnote{\url{https://rubocop.org/}}, ... If one assumes that a viable tool is available for every language that is worth augmenting, this more or less guarantees proper formatting results as we can assume that the most popular of these tools are tested and maintained by a large community.

\subsubsection{Pretty Printing}

But doing a \texttt{String -> String} transformation seems rather wasteful if the original syntax tree is still available. \cite{wadler_prettier_printer} describes a set of line-oriented operations that may be used to transform any AST into a tree of text nodes which are in turn well suited to be turned into a string.

\section{Generating IDEs}

And technically this dissertation does not describe only a compiler to transform an AST into source code, but also a second compiler to transform a Grammar into one of two IDE backends.

\cite{fraser_ten_2015} provides insight on visual cues that have been identified during the development of Blockly.

Wichtig: Turnaround-Zeit minimieren.

\subsection{Generation: BlattWerkzeug}

\subsection{Generation: Blockly}

\subsection{Runtimes}

\section{Transformations}

\subsection{Insertion Location}

\cite{fraser_ten_2015} describes \enquote{Transitive Connections} and doesn't like the idea of moving things after they have been dropped.

\subsection{Copying vs Moving}

\subsection{Appending vs Overwriting}

\subsection{Embracing}

\subsection{Derived values}

\subsection{Typed Holes}

Missing branches in the tree can be seen as typed holes as known from languages like Haskell (which was inspired by Agda) \cite{jones_haskell_2014} or Idris \cite{brady_type-driven_2017}. Filling these holes in real world programming languages is a difficult task, because narrowing down the vast number of choices to a \textit{reasonable} choice requires much context. Proposals to fill holes have been made for Haskell \cite{gissurarson_suggesting_2018} and attempts to create function implementations from a type signature and metadata like the name or a free form comment are under active research (e.g. \cite{chen_evaluating_2021} which is one of the parts that powers the currently not publicly available GitHub CoPilot\footnote{\url{https://copilot.github.com/}}.

\subsection{Internal Syntaxtree References}

\subsection{External Syntaxtree References}

\section{Practical Examples}

\subsection{XML}

\subsection{Mathematical Expressions and Algebraic Transformations}

\subsection{JavaScript}

\subsection{Python}

\subsection{SQL}

\section{Practical implications of Grammar Ambiguity for User Interfaces}

\subsection{SQL Distinct}

With single child list, with two child lists or with a property.

\section{Self Hosting}

Defining languages via the generated IDE requires the definition of a grammar and a block language that describes a grammar and a block language. We will call these special languages \textit{Meta Grammar} and \textit{Meta Block Language}.

\subsection{Meta Grammar}

\subsection{Meta Block Language}

\section{Code Generation}
\label{sec:code-generation-detail}

\subsection{Testing}

Testing code generation is a relatively simple process: The Syntaxtree can be defined as a \texttt{JSON}-file and the endresult is normal text.

\section{Conclusion}


%%% Local Variables:
%%% mode: latex
%%% TeX-master: "thesis"
%%% End:
