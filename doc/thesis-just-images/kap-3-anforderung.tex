\section{Anforderungsanalyse}
\label{sec:requirements}

In diesem Kapitel werden die Anforderungen beschrieben die an die
Containerisierung und die Bildverwaltung gestellt wurden.

Neben der Bildverwaltung stand auch die Verbesserung der Zugänglichkeit der
Hosting und Entwicklungsumgebung. Docker ist eine verbreitete Lösung um
Anwendungen in sogenannten Containern zu isolieren, die weitgehend
unabhängig von der verwendeten Hardware ist und auch auf den meisten
verbreiteten Betriebssystemen eingesetzt werden kann, ohne dass die auf dem
Hostsystem installierte zusätzlich installierte Software die Anwendung im
Container beeinflusst, sodass der Einfluss betriebssystemspeziefischer
Eigenheiten auf die Anwendung weitgehend vermieden werden kann.

Zur Bildverwaltung gehöhren ein API\todo[author=mri]{Typo, Satzbau},
die es ermöglicht, Bilder und essentielle Metadaten anzulegen,
auszugeben, zu bearbeiten und zu löschen, eine Bibliothek, die die
durch die API bereitgestellten Funktionen Serverseitig umsetzt und
eine Frontendintegration, die dem Verwalter eines Projekts das
Anlegen, Betrachten, Modifizieren und Löschen der Bilder und Metadaten
ermöglicht sowie die Verwendung der Bilder in Projektvorschau, dem
Seiteneditor und den Datenbanktabellen.

\subsection{Containerisierung}

Das Projekt verwendet drei Ausprägungen von Images, jeweils eins für
Produktivumgebung, Entwicklungsumgebung und Testumgebung.

\todo[author=mri,inline]{Begründung hier, Erläuterung in den
  Sektionen. Verweis auf Buildservices wie Travis oder Bitbucket
  Pipelines}

\subsubsection{Entwicklungsumgebung}

Das Entwicklungsimage stellt die während der Softwareentwicklung
benötigte Umgebung zur Verfügung und erwartet das Einbinden des
Quellcodes via Volume, sodass eine Änderung am Quellcode auf dem Hostsystem
sofort im Container sichtbar wird und zum neucompilieren des Codes durch den im
Entwicklungsmodus laufenden Server führt. So ist es möglich auf einem
Entwicklungsgerät zu arbeiten ohne die Abhängigkeiten im Betriebssystem
zu installieren und somit inkompatible Versionen von Abhängigkeiten mit
anderen Projekten zu vermeiden. Weiterhin ist es so möglich unter Windows
und MacOS zu entwickeln, da hier seitens Docker Werkzeuge zum Ausführen von
Containern mittels der Linux Distrubution boot2docker in einer sehr
minimalistischen durch die Werkzeuge verwalteten virtuellen Maschiene.

\todo[author=mri,inline]{Nicht erwähnter Vorteil: Entwickler kann
  seine gewohnte Umgebung weiternutzen}

\subsubsection{Testumgebung}

Das Testimage weist Ähnlichkeiten zum Entwicklungsimage auf, enthält
aber noch den Chromium Browser, der im Headless Mode für Tests des Clients
genutzt wird.

\subsubsection{Produktivumgebung}

Das Produktivimage enthält eine fertig gebaute Version der Software sowie die
zum Betrieb nötigen Abhängigkeiten, die Versionierung des Images richtet sich
nach der Version der Software, ein Update wird durch das herunterladen/bauen des
Images mit der entsprechend höheren Version und austauschen des laufenden
Containers mit einem Container der neueren Version realisiert.

\subsection{Bildverwaltung}

Der erste zu klärende Punkt der Bilddatenbank war, wie Bilder zur
Datenbank hinzugefügt werden können. Zur
Auswahl\todo[author=mri]{Auflistung} standen der Dateiupload im
Projekteditor durch den Autoren des Projekts, Import aus einer
öffentlichen Bilddatenbank, Import aus einer eigenen durch den
Administrator gepflegten Bilddatenbank.  Eine dieser Möglichkeiten
soll mit der Implementierung der Bildverwaltung direkt umgesetzt
werden, die anderen möglicherweise später folgen.

Vorteile\todo[author=mri]{Auch Tabellarisch} der öffentlichen
Bilddatenbank ist der geringe Aufwand beim Import von Bildern, bei
ausschließlicher Nutzung von Datenbanken die alle Bilder unter
Creative Commons Zero (CC0) veröffentlichen, wie beispielsweise
openclipart, entfällt zudem die Notwendigkeit Metadaten wie Urheber
und Lizenz zu Speichern oder anzuzeigen, bei Datenbanken die auch
andere Creative Commons Lizenzen (CC-BY, CC-BY-SA,CC-BY-NC,
CC-BY-SA-NC, CC-BY-ND, CC-BY-ND-NC) oder vergleichbare zulassen sind
Urheber und Lizenz zumindest Maschienenlesbar hinterlegt, sodass im
einfachsten Fall die Auswahl der Bilddatenbank und die ID des Bildes
für den Import ausreichen, im angenehmen Fall ist im Projekts Editor
ein Suchinterface mit Anbindung an die API der Bilddatenbank sodass
eine oder alle Bilddatenbanken aus dem Projekteditor heraus durchsucht
werden können ohne dass dem Projektautoren die jeweilige Bilddatenbank
bekannt sein muss.

Die Nachteile der öffentlichen Bilddatenbanken als initial einziges Verfahren
zum hinzugefügen von Bildern ist die Verwendung eigener Werke. Ein eigenes Werk
müsste erst einer angebundenen Bilddatenbank hinzugefügt werden um es verwenden
zu können. Dies erfordert zum einen die Bereitschaft des Urhebers das Werk unter
einer der von der Bilddatenbank erlaubten Lizenz zu veröffentlichen, sowie die
Eröffnung eines Accounts bei der jeweiligen Bilddatenbank. Da es sich bei der
Zielgruppe von SQLino um Schüler handelt, die auch zum Teil zu jung sind um
Verträge wie den zur Eröffnung eines Accounts einzugehen, wodurch sie von der
Verwendung eigener Werke ausgeschlossen sind bis der Dateiupload implementiert
ist, weshalb diese Importmethode als vorerst einzige ausscheidet.

\todo{warum die admindatenbank nicht genommen wird: kann auch aus oeffentlichen datenbanken importieren,
  freigabeprozess fuer eigene werke, vorauswahl, hoher aufwand}

Die zusammen mit der Bildverwaltung zu implementierende Variante ist daher der
Dateiupload im Projekteditor. Der Import von Bildern aus Bilddatenbanken ist so,
zwar unter höherem Aufwand, möglich, und durch manuell anzugebenden Urheber und
Lizenz können Schüler so für die Notwendigkeit der Auseinandersetzung mit
Urheberrecht bei Veröffentlichung von Bildern sensibilisiert werden.

\subsubsection{Lizenzen \& Metadaten}

Da Bilder nicht alle unterstützenswerten Bildformate Metadaten über
Urheber und Lizenz standatisiert haben, ist es nicht möglich die aus
rechtlicher Sicht Notwendigen Metadaten zum Urheber und zur Lizenz der Bilder
in der Datei selbst zu speichern und aus rechlicher Sicht, reicht die bloße
Anwesenheit der Daten in der Datei nicht um Beispielsweise die Anforderung der
Urhebernennung zu erfüllen, weshalb die aus rechlicher Sicht notwendigen
Metadaten in einer zusätzlichen Datenquelle gehalten werden müssen.

Die aus rechtlicher Sicht notwendigen Daten sind Name des Urhebers mit Link auf
eine Webpräsenz und der Name der Lizenz unter der das Bild verwendet wird
mit Link auf die Lizenz.

\subsubsection{Darstellungsformen}

Für die Bilder sollen zwei Darreichungsformen existieren: Die Abbildung
(Figure) und das gestalterische Element (Graphic), die sich durch ihre
Einbettung und den Ort der Quellenangaben unterscheiden. Die Abbildung
eingerahmt, die Metadaten am Bild dargestellt, das Gestalterische Element bettet
das Bild in einen Link ein, der auf die Quellenangabe am Ende der Seite verweist.

\subsubsection{Darstellung von Quellenangaben}

Bei Abbildungen werden Name des Bildes, Urheber des Bildes und Name der Lizenz
direkt unter dem Bild innerhalb des eingerahmten Bereichs dargestellt, der Name
des Urhebers ist ein Link auf die Webpräsenz, der Name der Lizenz ein Link
auf den Volltext der Lizenz.

Die Quellenangaben der gestalterischen Elemente findet tabellarisch im Footer der
Webseite statt, Alle dargestellten gestalterischen Elemente werden dort in der
Reihenfolge ihres Erscheinens im Quelltext der Seite aufgelistet. Die Tabelle
besteht aus den Spalten Nummer des Quellverweises, Name des Bildes,
Name des Urhebers und Name der Lizenz, wobei die Nummer des Quellverweises auf
das gestalterische in der Seite verlinkt, der Name des Bildes auf eine neue
Seite bestehend aus einer Abbildung des Bildes gemäß der obig genannten
Darstellungsform von Abbildungen, der Name des Autoren ein Link auf dessen
Webpräsenz und der Name der Lizenz ein Link auf den Volltext der Lizenz.

\subsubsection{Notwendigkeit von Quellenangaben}

Sofern notwendig kann die obig genannte Darstellung der Quellenangaben nicht
weiter beeinflusst werden als durch die Auswahl eines der genannten Typen. Liegt
jedoch eine Grafik vor, deren Lizenz weder die Namensnennung noch die Nennung
der Lizenz erfordert, wie beispielsweise CC0, oder der Autor der Seite selbst der
Urheber ist, soll auf die Quellenangabe verzichtet werden können, im Falle
einer Abbildung also nur der Name angezeigt werden und im Falle des
gestalterischen Elements der Eintrag in der Liste komplett weggelassen werden.

\subsubsection{Endgerätoptimierung}

\todo{}

\subsubsection{Datenhaltung}

Die Datenhaltung betrifft sowohl die Bilder selbst als auch die Metadaten. Die
Bilder selbst können entweder in einem Key-Value Store oder im Dateisystem
gespeichert werden. Im Fall der Bilder muss das verwendete verwendete System in
der Lage sein mit großen Blöcken von Binärdaten effizient umzugehen. In
beiden Fällen müssen die Operationen Einfügen, Löschen, Indizierter Zugriff und
im Falle der Metadaten auch die Komplettzugriff auf alle Werte nach Möglichkeit
in Abhängigkeit von der Anzahl der angefragten/bearbeiteten Elemente komplex
sein und nicht in Abhängigkeit der insgesammt gespeicherten Elemente. Daher sind
an dieser Stelle datenbankähnliche Systeme, die beim einfügen und löschen von
Daten ihren Index neu berechnen müssen, der für die gute Suchperformance sorgt,
wie beispielsweise Elasticsearch unbrauchbar, auch da eine Suche im
Funktionsumfang der Bildverwaltung nicht vorgesehen ist. Der Key-Value Store
muss weiterhin die für die unterscheidung der Bilder verwendeten UUIDs als Key
unterstützten. Die bisher verwenden sqlite Datenbanken sind nicht für den
Einsatz als Key-Value Store mit Values von der Größe Bildern konzipiert,
vergrößern die Datei in der die Datenbank persistiert wird und erhöhen somit die
benötigte Menge an Arbeitsspeicher um die Datenbank darin zu laden um
Operationen darauf auszuführen. Es müsste also ein zusätzlicher Dienst in die
Projektarchitektur eingefügt werden, um die Funktionalität des Key-Value Stores
zu übernehmen. Da Ein Wechsel weg von Sqlite und hin zu PostgreSQL ist zunächst
die Tauglichkeit von PostgreSQL als Key-Value Store zu betrachten und im Falle
der Tauglichkeit anderen Dienstbasierten Lösungen vorzuziehen.

PostgreSQL kennt eine Key-Value Store implementierung namens hstore
\cite{PostgreSQL-hstore} bei der Key-Value Tupel gespeichert werden. Desweiteren
unterstützt PostgreSQL von großen und sehr großen Blöcken von Binärdaten, es
wird jedoch davon Abgeraten sehr große Dateien deren Zugriffszeit kritisch für
die Anwendung ist, da der Zugriff aus der Datenbank weniger Performant als der
direkte Zugriff auf das Dateisystem \cite{PostgreSQL-blob}. Außerdem erhöht Last
auf der Datenbank die Zugriffszeiten auf die Bilder als auch das Anzeigen vieler
Bilder die Last auf der Datenbank. Desweiteren ist eine Verteilung von Datenbank
und Bildern auf verschiedene Systeme (Content Delivery Network) nicht ohne
verteilte Datenbank realisierbar und weniger Performant. Ab welcher
Größenordnung dies zu einem Problem wird müsste in einem Benchmark getestet
werden bei dem Diverse Nutzer auf Diversen Projekten simuliert werden müssten,
wofür erst die PostgreSQL Anbindung abgeschlossen werden müsste.

Letztere Beeinträchtigungen wären bei einem dedizierten Key-Value Store nicht
gegeben, führen aber zu mehr Komplexität in der Systemarchitektur und somit zu
mehr Fehlerquellen. \todo{BerkeleyDB und memcacheDB anreissen}

Das Speichern der Bilder im Dateisystem unter Verwendung der UUID zum Zugriff
ist so perfomant, wie Speichermedium und Dateisystem es zulassen. Bei aktuellen
Dateisystemen wie ext4 hat die Anzahl der Dateien in einem Verzeichnis keinen
nennenswerten Einfluss auf die Zugriffszeiten auf einzelne bekannte Dateien,
lediglich die des Auflisten aller Dateien ist von der Anzahl linear abhängig.
Sollten die Zugriffe auf die Bilder je die Kapazität des Webservers
überschreiten, so kann bei dieser Methode einfach ein Content Delivery Network
eingebunden werden indem Bilder via NFS und Caching von anderen Servern
ausgeliefert werden oder einem CDN-Dienstleister synchronisiert werden.
Desweiteren kann der Administrator ein Bild einfacher anhand seiner UUID finden
und durch einene Löschnotiz ersetzen, falls Urheberrecht oder Strafrecht dies
erfordern, da die Datei weder gesucht werden muss, noch ein Client zum
betrachten und bearbeiten der Datenbank sowie Kentnisse darüber notwendig sind;
die zur Administration eines Servers notwendigen Kentnisse sind ausreichend.



%%% Local Variables:
%%% mode: latex
%%% TeX-master: "thesis"
%%% End:
