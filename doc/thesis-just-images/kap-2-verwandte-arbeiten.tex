\section{Verwandte Arbeiten und Produkte}
\label{sec:related-work}

\subsection{TODO: Docker}
Docker verbindet das Isolationsprintip der BSD Jails mit einer Deployment
Strategie und Aufeinander aufbauenden Environments unter Linux.
Es kommt hier ein OverlayFS zusammen mit einem Chroot zum Einsatz, um das
Isolierte Dateisytem zu generieren. Weiterhin werden Virtuelle Netzwerkadapter
und ein Virtueller Prozess ID Raum erstellt. Der Kernel des Hostsystems wird mit
dem Container geteilt, f\"{u}r Prozesse des Hostsystems sind alle in Containern
laufenden Prozesse sowie alle virtuellen Netzwerkadapter und auch die via
OverlayFS angelegten Verzeichnisstrukturen, f\"{u}r Prozesse des Containers
sieht es aus, als handele es sich um den init Prozess auf einem eigenen
Computer.

Ein Projekt wird in Services aufgeteilt, wie z.B. Reverse-Proxy, Webapp,
API-Server, Datenbank und Static-File-Server, die dann in Containern isoliert
laufen, wobei jeder Container ein Virtuelles Environment darstellt, in dem 
jeweils nur ein Prozess mit eventuell geforkten Worker Prozessen l\"{a}uft.

Da ein Container mit Ausnahme der Hardware und der Kernel Version \"{u}berall
das gleiche Environment erzeugt, beinhaltet Docker eine Deployment Strategie
mittels Images. Ein Image verh\"{a}lt sich dabei zu einem Container in etwa so
wie eine Klasse zu einer Instanz bei Objektorientierter Programmierung und
ebenso, kann ein Image von einem anderen Image erben. Ein Image wird durch den
Build Prozess aus einem Dockerfile erzeugt und ist unveraenderlich. Das
Dockerfile ist eine Anleitung um aus einem Image ein anderes zu bauen und
beginnt mit der Nennung des Basisimages. Das immer existierende Basisimage
scratch stellt ein vollkommen leeres Dateisytem bereit und ist am Anfang jeder
Vererbungskette zu finden. Jede weitere Anweisung im Dockerfile f\"{u}gt dem
Image einen weiteren Layer hinzu, der \"{A}nderungen vorheriger Layer mittels
OverlayFS \"{u}berlagert. Insbesondere bei Produktivumgebungen sollte versucht
werden, die Anzahl der Layer durch zusammenlegen von Anweisungen im Dockerfile
zu minimieren, und somit den Overhead durch das OverlayFS beim Dateizugriff zu
minimieren.

Die Einzelnen Layer werden in Form komprimierter Archive adressiert nach
Hashwerten gespeichert, die beim Erzeugen eines Containers entpackt werden und
so den Startzustand des Containers herstellen. Alle im Container gemachten
\"{A}nderungen, die nicht zu einem Image Layer gepackt oder in einem Volume
gespeichert werden gehen beim entfernen des Containers verloren. Volumes sind
Verzeichnisse im Hostsystem, die an eine bestimmte Stelle des Containers gemappt
werden und gleichzeitig in mehreren Containern eingebunden sein k\"{o}nnen.

Eingehende Kommunikation zu Containern wird per Default unterbunden und muss
explizit aktiviert werden. Es ist m\"{o}glich mehrere Container in ein
gemeinsames Virtuelles Netzwerk zu legen, sodass eine Kommunikation
untereinander stattfinden kann (Reverse-Proxy -> Webseite, Webseite ->
Datenbank) als auch bestimmte Netzwerk Ports des Hosts auf bestimmte Netzwerk
Ports eines Containers zu mappen (Internet -> Reverse-Proxy).

Das Projekt verwendet drei Auspr\"{a}gungen von Images, jeweils eins f\"{u}r
Produktivumgebung, Entwicklungsumgebung und Testumgebung. Das Produktivimage
enth\"{a}lt eine fertig gebaute Version der Software sowie die zum Betrieb
n\"{o}tigen Abh\"{a}ngigkeiten, die Versionierung des Images richtet sich nach
der Version der Software, ein Update wird durch das herunterladen/bauen des
Images mit der entsprechend h\"{o}heren Version und austauschen des laufenden
Containers mit einem Container der neueren Version realisiert.

Das Entwicklungsimage stellt die w\"{a}hrend der Softwareentwicklung
ben\"{o}tigte Umgebung zur Verf\"{u}gung und erwartet das Einbinden des
Quellcodes via Volume, sodass eine \"{A}nderung am Quellcode auf dem Hostsystem
sofort im Container sichtbar wird und zum neucompilieren des Codes durch den im
Entwicklungsmodus laufenden Server f\"{u}rt. So ist es m\"{o}glich auf einem
Entwicklungsger\"{a}t  zu arbeiten ohne die Abh\"{a}ngigkeiten im Betriebssystem
zu installieren und somit inkompatible Versionen von Abh\"{a}ngigkeiten mit
anderen Projekten zu vermeiden. Weiterhin ist es so m\"{o}glich unter Windows
und MacOS zu entwickeln, da hier seitens Docker Werkzeuge zum Ausf\"{u}hren von
Containern mittels der Linux Distrubution boot2docker in einer sehr
minimalistischen durch die Werkzeuge verwalteten virtuellen Maschiene.

Das Testimage weist \"{A}hnlichkeiten zum Entwicklungsimage auf, enth\"{a}lt
aber noch den Chromium Browser, der im Headless Mode f\"{u}r Tests des Clients
genutzt wird.

\subsection{TODO: Short description of SQLino}

\subsection{TODO:RESTFUL API}
Eine RESTFUL API ist eine, die zustandsfreie HTTP Requests verwendet. Ein API
Endpunkt wie z.B. https://example.com/foo/23/bar/42 wird verwendet, indem ein
Request mit einem der HTTP Verben GET, POST, PUT oder DELETE gesendet wird.
GET signalisiert die Abfrage von Daten, POST das Ver\"{a}ndern von Daten, PUT
das Anlegen neuer Daten und DELETE das L\"{o}schen von Daten. Ein GET Request
k\"{o}nnte beispielsweise ein bestimmtes Bild abrufen, ein anderer eine
Datenbank nach allen Eintr\"{a}gen, die einem mitgeschicktem Filterkriterium
entsprechen, durchsuchen. Es ist auch vorstellbar ein RESTFUL API auf ein
definiertes Subset von SQL zu mappen oder die Steuerung einer Waschmaschiene.

Da die Requests zustandsfrei sind, kann in einem Request nicht auf einen
vorausgegangenen Request zur\"{u}ckgegriffen werden. Zwar werden duch einen
Request eingetretene Datenbank\"{a}nderungen in der Datenbankabfrage des
n\"{a}chsten Requests sichtbar, allerdings, existiert keine Session in der
beispielsweise ein Arbeitsverzeichnis festgelegt oder ein von allen
Folgerequests verwendeter Namensraum ausgew\"{a}lt werden kann. Jeder Request
muss alle zu seiner Verarbeitung notwendigen Informationen selbst beinhalten. 

\subsection{TODO: MVC}
Model View Controller ist ein zum bearbeiten von Webrequests geeignetes Konzept.
Daten werden nach durch ein Model beschrieben und strukturiert und verwaltet, ihre Darstellung
durch den View festgelegt und die Webrequests vom Controller behandelt. So
werden Datenhaltung, Darstellung und Buisiness Logic von einander getrennt und
k\"{o}nnen bis zu einem gewissen Grad unabh\"{a}ngig von einander
\"{u}berarbeitet werden.

\subsection{TODO: Short description of Angular}

%%% Local Variables:
%%% mode: latex
%%% TeX-master: "thesis"
%%% End:
