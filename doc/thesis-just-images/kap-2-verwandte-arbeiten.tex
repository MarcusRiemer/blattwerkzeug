\section{Verwandte Arbeiten und Produkte}
\label{sec:related-work}

\todo[author=mri,inline]{Kurz den Zweck des Kapitels umreißen, keine
Überschriften ohne Text!}

\subsection{Docker} Docker verbindet das Isolationsprintip der BSD
Jails mit einer Deployment Strategie und Aufeinander aufbauenden
Environments unter Linux.  Es kommt hier ein OverlayFS zusammen mit
einem Chroot zum Einsatz, um das Isolierte Dateisytem zu
generieren. Weiterhin werden Virtuelle Netzwerkadapter und ein
Virtueller Prozess ID Raum erstellt. Der Kernel des Hostsystems wird
mit dem Container geteilt, für Prozesse des Hostsystems sind alle in
Containern laufenden Prozesse sowie alle virtuellen Netzwerkadapter
und auch die via OverlayFS angelegten Verzeichnisstrukturen, für
Prozesse des Containers sieht es aus, als handele es sich um den init
Prozess auf einem eigenen Computer.

Ein Projekt wird in Services aufgeteilt, wie z.B. Reverse-Proxy,
Webapp, API-Server, Datenbank und Static-File-Server, die dann in
Containern isoliert laufen, wobei jeder Container ein Virtuelles
Environment darstellt, in dem jeweils nur ein Prozess mit eventuell
geforkten Worker Prozessen läuft.

\todo[author=mri,inline]{Grafik zur Aufteilung in Services}

Da ein Container mit Ausnahme der Hardware und der Kernel Version
überall das gleiche Environment erzeugt, beinhaltet Docker eine
Deployment Strategie mittels Images. Ein Image verhält sich dabei zu
einem Container in etwa so wie eine Klasse zu einer Instanz bei
Objektorientierter Programmierung und ebenso, kann ein Image von einem
anderen Image erben. Ein Image wird durch den Build Prozess aus einem
Dockerfile erzeugt und ist unveraenderlich\todo[author=mri]{Umlaute
sind OK}. Das Dockerfile ist eine Anleitung um aus einem Image ein
anderes zu bauen und beginnt mit der Nennung des Basisimages. Das
immer existierende Basisimage scratch stellt ein vollkommen leeres
Dateisytem bereit und ist am Anfang jeder Vererbungskette zu
finden. Jede weitere Anweisung im Dockerfile fügt dem Image einen
weiteren Layer hinzu, der Änderungen vorheriger Layer mittels
OverlayFS überlagert. Insbesondere bei Produktivumgebungen sollte
versucht werden, die Anzahl der Layer durch zusammenlegen von
Anweisungen im Dockerfile zu minimieren, und somit den Overhead durch
das OverlayFS beim Dateizugriff zu minimieren.

Die Einzelnen Layer werden in Form komprimierter Archive adressiert
nach Hashwerten gespeichert, die beim Erzeugen eines Containers
entpackt werden und so den Startzustand des Containers
herstellen. Alle im Container gemachten Änderungen, die nicht zu einem
Image Layer gepackt oder in einem Volume gespeichert werden gehen beim
entfernen des Containers verloren. Volumes sind Verzeichnisse im
Hostsystem, die an eine bestimmte Stelle des Containers gemappt werden
und gleichzeitig in mehreren Containern eingebunden sein können.

Eingehende Kommunikation zu Containern wird per Default unterbunden
und muss explizit aktiviert werden. Es ist möglich mehrere Container
in ein gemeinsames Virtuelles Netzwerk zu legen, sodass eine
Kommunikation untereinander stattfinden kann (Reverse-Proxy ->
Webseite, Webseite -> Datenbank) als auch bestimmte Netzwerk Ports des
Hosts auf bestimmte Netzwerk Ports eines Containers zu mappen
(Internet -> Reverse-Proxy).

\todo[author=mri,inline]{Weitere Grafik oder auf die bisherhige Grafik
verweisen}

\subsubsection{Docker vs VM}

\todo[author=mri,inline]{TODO}

\subsection{TODO: Short description of SQLino}

\subsection{TODO:RESTFUL API} Eine RESTFUL API ist eine, die
zustandsfreie HTTP Requests verwendet. Ein API
Endpunkt\todo[author=mri]{Stichwort: \texttt{URI}} wie
z.B. https://example.com/foo/23/bar/42 wird verwendet, indem ein
Request mit einem der HTTP Verben GET, POST, PUT oder DELETE gesendet
wird.  GET signalisiert die Abfrage von Daten, POST das Verändern von
Daten, PUT das Anlegen neuer Daten und DELETE das Löschen von
Daten. Ein GET Request könnte beispielsweise ein bestimmtes Bild
abrufen, ein anderer eine Datenbank nach allen Einträgen, die einem
mitgeschicktem Filterkriterium entsprechen, durchsuchen. Es ist auch
vorstellbar ein RESTFUL API auf ein definiertes Subset von SQL zu
mappen oder die Steuerung einer Waschmaschiene.

Da die Requests zustandsfrei sind, kann in einem Request nicht auf
einen vorausgegangenen Request zurückgegriffen werden. Zwar werden
duch einen Request eingetretene Datenbankänderungen in der
Datenbankabfrage des nächsten Requests sichtbar, allerdings, existiert
keine Session in der beispielsweise ein Arbeitsverzeichnis festgelegt
oder ein von allen Folgerequests verwendeter Namensraum ausgewählt
werden kann. Jeder Request muss alle zu seiner Verarbeitung
notwendigen Informationen selbst beinhalten.

\subsection{TODO: MVC} Model View Controller ist ein zum bearbeiten
von Webrequests geeignetes Konzept.  Daten werden nach durch ein Model
beschrieben und strukturiert und verwaltet, ihre Darstellung durch den
View festgelegt und die Webrequests vom Controller behandelt. So
werden Datenhaltung, Darstellung und Buisiness Logic von einander
getrennt und können bis zu einem gewissen Grad unabhängig von einander
überarbeitet werden.

\subsection{TODO: Short description of Angular}

%%% Local Variables:
%%% mode: latex
%%% TeX-master: "thesis"
%%% End:
