\section{Implementierung}
\label{sec:implementation-analysis}

\subsection{Kontext}



\subsubsection{Kommunikation via Description Interface}



\subsubsection{Server: Ruby on Rails}



\subsubsection{Client: Angular 4}



\subsubsection{Caching}

\todo{caching von html requests}

\subsubsection{API}

Die API stellt verschiedene Routen bereit, die sich in das existierende
Strukturen einfügen, daher sind alle Routen unterhalb von
\begin{description}
\item [/api/project/:project\_id/image]
\end{description}
angesiedelt

\begin{description}
  \item [/api/project/:project\_id/image]
  \item [GET]
  \item [POST]
\end{description}
\begin{description}
  \item [/api/project/:project\_id/image/:image\_id]
  \item [GET]
  \item [POST]
  \item [DELETE]
\end{description}

\subsubsection{Library}



\subsubsection{Frontendintegration}



\subsubsection{Seiteneditorintegration}



\subsubsection{Datenbankintegration}



\subsubsection{Schemaeditorintegration}



\subsubsection{Queryeditorintegration}



%%% Local Variables:
%%% mode: latex
%%% TeX-master: "thesis"
%%% End:
