\section{Fazit}
\label{sec:conclusion}

Die in Kapitel \todo{Verweis} beschriebenen Ziele wurden grundsätzlich erreicht:
Schüler können neue Bilder in Ihre Projekte einpflegen (und diese auch verwenden).
Außerdem wurden die Kompilierungsschritte in einer weitestgehend
platformunabhängigen Docker-Umgebung gekapselt. Dieses Kapitel fasst zusammen
inwiefern die jeweiligen Implementierungen vollständig sind und gibt einen Ausblick
auf mögliche Erweiterungen.

\subsection{Erreichte Ziele}

Es wurde eine einheitliche Umgebung geschaffen um die Rails und Angular
Toolchain zu Betreiben. Wie in Kapitel \nameref{subsec:4-containerization} erläutert,
wurde eine Umgebung für die Verwendung durch Entwickler zum ausführen der
Anwendung im Entwicklungsmodus sowie eine zum automatisierten Ausführen der
Tests.

Weiterhin wurde wie in Kapitel \nameref{subsec:4-image-library} ff. ein Prototyp einer
Bildverwaltung implementiert, der es Erlaubt dem Projektautoren Bilder für die
Nutzung innerhalb des Projekts hochzuladen und sowohl als statische Inhalte im
Seiteneditor zu verwenden, als Ergebnis von Datenbankabfragen zu rendern. Bilder
die als Ergebnis einer Datenbankabfrage gerendert werden, werden im Footer der
Seite mit Urheber und Lizenz aufgeführt, sowie alle nicht als Figure Umgebung
angezeigten statischen Bilder. Bei Figure Umgebungen werden die Angaben zu
Urheber und Lizenz zusammen mit dem Bildtitel direkt unter dem Bild angezeigt.

\todo[inline]{bild von einer seite, die dies demonstriert}

\subsection{Nicht erreichte Ziele}

Die Urprünglichen Ziele konnten aufgrund einiger unerwarteter Untiefen im Umfang
der Umsetzung nicht erreicht werden. Offen geblieben sind
\begin{easylist}[itemize]
  & Das Durchsuchen mit Import von öffentlichen Bilddatenbanken
  & Die Projektübergreifende Bilddatenbank, die diverse generischen Bilder
  bereitstellt
  & Ein Interface für den Systemadministrator/Lehrer um Projektübergreifend nach
  Bildern zu suchen und diese zu sperren.
  & Die Definition von zuätzlichen Bildeigenschaften im Seiteneditor, um die
  Seitengestaltung zu verbessern
  & Die vereinheitlichte Umgebung für den Produktivbetrieb von SQLino
\end{easylist}

\todo[inline]{Wie vorher: Description-Umgebung mit den gekürzten Listenelementen als Überschrift}

\nameref{subsec:3-image-library} geht darauf ein, weshalb der Bildupload pro
Projekt höher priorisiert wurde, als die nicht umgesetzen Varianten des Imports
aus öffentlichen Bilddatenbanken und der Zentralen Bilddatenbank. Natürlich ist
die Umsetzung aller drei wünschenswert, muss aus zeitlichen Gründen aber später
erfolgen.

Das Interface zum Bilder sperren ist zwar notwendig, bevor SQLino produktiv
genutzt werden kann, setzt allerdings die vorherige Umsetzung der erreichten
Ziele voraus, weshalb diese vorrangig behandelt wurden.

Wie im Kapitel \nameref{subsec:4-page-editor} dargelegt konnten nicht alle Features für
den Seiteneditor umgesetzt werden, weshalb dort nur ihr Sollzustand beschrieben
ist. Die aktuelle Implementierung erlaubt es nicht, die Dimensionen der
dargestellten Bilder im Seiteneditor festzulegen, daher werden alle Bilder immer
mit maximaler Größe dargestellt. Diese ergibt sich aus der Breite des umgebenden
Elements der Seite sowie bei Rastergrafiken aus der Auflösung, in der das Bild
vorliegt, daher ist die derzeit einzige Möglichkeit ein Bild kleiner als über
die gesammte Breite der Seite darzustellen die Verwendung der \texttt{row} und
\texttt{col} Elemente des Seiteneditors zur Beschränkung der Bildbreite. Derzeit
ist es nicht möglich beispielsweise Pixelart groß darzustellen, ohne sie mit
einem Grafikeditor zu skalieren und der Bildverwaltung in dieser Version
hinzuzufügen. Weiterhin fehlt die Möglichkeit ein Bild von Text umfließen zu
lassen.

Da SQLino noch nicht den Stand des Produktivbetriebs erreicht hat, war die
Priorität der einheitlichen Produktivumgebung, die eine direkt lauffähige
Instanz von SQLino bereitstellt hinten angestellt und dementsprechend nicht
vollendet. Allerdings sind diverse dafür notwendige Schritte analog zu der
Entwicklungs und Testumgebung, allerdings mit anderer Konfiguration.

\subsection{Weiterentwicklung}

\todo[inline]{TODO was aus der liste von nicht erreichte ziele muss eigentlich
  hier stehen?

  Anpassungen nach der Umstellung auf PostgreSQL

  Anpassungen nach dem Implementieren der Projektvererbung
}



%%% Local Variables:
%%% mode: latex
%%% TeX-master: "thesis"
%%% End:
