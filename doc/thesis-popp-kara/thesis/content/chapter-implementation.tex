%************************************************
% Implementierung
%************************************************
\chapter{Implementierung}
\label{sec:implementation}

\section{Objektstruktur}

- World, WorldState, Truck, Tile, TrafficLight
- UML-Diagramm

\section{Programmiersprache}

- Befehle auf Deutsch
- Zwei Arten von Schleifen
- Definition von Funktionen und Schleifenvariablen
- Befehle werden auf einer Welt ausgeführt
- Kompilieren in JS-Code mit async await
- new AsyncFunction (TypeScript-Compiler-Problem, Anforderung an den Browser)

\section{Darstellung}

\subsection{Technologie}

- Darstellung mit SVG und deren Animationen gut in Angular integriert
- Angular-Animationen nutzen ein State-System, mit dem sich nur sehr umständlich
  die benötigten Animationen umsetzten lassen (keine festen Zustände, sondern
  variable Positionen)
- Erster Prototyp mit Canvas, dann Umstieg auf Canvas
- Canvas keine direkte Integration in Angular
- Canvas-Renderer läuft außerhalb der Angular-Zone
- WebGL raus weil Nutzen-Aufand nicht gerechtfertigt

\subsection{Lastwagen-Position}

- Möglichkeit A: Lastwagen halten auf Mitte der Kachel und drehen sich auf der
                 Stelle => Lastwagen können auch wenden, aber das Verhalten
                 sieht unnatürlich aus.
- Möglichkeit B: Lastwagen halten auf dem Rand der Kachel und es kann ein
                 Blinker nach links oder rechts gesetzt werden, der die
                 Fahrtrichtung beim nächsten Vorwärtsfahren beeinflusst
                 => Verhalten sieht deutlich natürlicher aus und ist näher am
                 Vorbild. Allerdings kann der Lastwagen nun nicht mehr auf der
                 Stelle wenden und muss eventuell längere Strecken fahren.
                 => Erste Idee: Pfeil davor anzeigen, Zweite Idee: Blinker
                 => Rückwärts fahren?

\subsection{Animationsschritte}

- Neuer Zustand für jede Operation
- Zustände haben unterschiedliche Zeitwerte: Blinker setzen kostet keine Zeit,
  geradeaus fahren oder warten kostet einen Zeitschritt
- Zeitschritte sind besonders für die Berechnung der Ampelstellung relevant
- Position des Lastwagen wird schrittweise interpoliert

- Schitte haben eine feste Zeit, die aber übersprungen wird, wenn ein neuer
  Schritt vor ablauf der Zeit dazu kommt. Dadurch kann ein Programm auch im
  "Schnelldurchlauf" abgespielt werden.

Ansätze "absolute" vs "relative" Bewegung gegenüberstellen

\section{Integration}

- Zunächst Prototypenentwicklung auf "Grüner Wiese", dann Integration in das
  bestehende Projekt

\subsection{Angular-Service}

- Auslagerung der Welt in einen Service

\subsection{Fortschrittsmeldungen}

- Rückmeldung über den Fortschritt des Programm zur Anzeige im Drag \& Drop-Editor
