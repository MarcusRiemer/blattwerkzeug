%************************************************
% Grundlagen
%************************************************
\chapter{Grundlagen}
\label{sec:basics}

In diesem Kapitel werden einige Grundlagen vermittelt, auf welche in späteren Teilen dieser Arbeit verwiesen wird. Aus den Abschnitten "\nameref{sec:basics:playful-learning}" und "\nameref{sec:basics:mini-languages}" werden zudem Anforderungen an diese Arbeit abgeleitet.

%************************************************
% Spielerisches Lernen
%************************************************
\section{Spielerisches Lernen}
\label{sec:basics:playful-learning}

Die im Rahmen dieser Arbeit entwickelte Software soll Lehrer dabei unterstützen, ihren Schülern die Grundlagen des Programmierens zu vermitteln. Dies soll spielerisch geschehen. Prensky führt in seinem Buch \textit{Digital Game-Based Learning} drei Gründe an, warum digitales spielerisches Lernen funktioniert~\cite[147]{prensky2007}:

\begin{enumerate}
    \item Der erste Grund ist das zusätzliche Engagement, das dadurch entsteht, dass das Lernen in einen Spielkontext gebracht wird. Dies kann beträchtlich sein, besonders wenn Menschen nicht lernen wollen.
    \item Der zweite Grund ist der interaktive Lernprozess. Dieser kann und sollte abhängig von den Lernzielen viele verschiedene Formen annehmen.
    \item Der dritte Grund ist die Art und Weise, wie die zwei ersten Gründe im Gesamtpaket zusammengefügt werden. Es gibt viele Möglichkeiten, dies zu tun, und die beste Lösung ist sehr kontextabhängig.
\end{enumerate}

Des Weiteren hängt der Lernerfolg auch immer stark davon ab, wie Spiele vom Lehrer letztendlich eingesetzt werden, aber auch der Stil des Spieles spielt eine Rolle. Damit Spiele -- Lehrspiele im speziellen -- Spaß bringen, müssen sie einige Anforderungen erfüllen. Malone stellt in seinem Artikel \textit{What Makes Computer Games Fun?} eine Checkliste auf, die sich in drei Kategorien gliedert und unter anderem die folgenden Fragen enthält~\cite[49]{malone1981}:

\begin{itemize}
    \item \emph{Herausforderung}: Hat das Spiel ein Ziel? Hat das Spiel einen variablen Schwierigkeitsgrad? Verfügt die Aktivität über mehrere Ziele, z.~B. Zählen von Punkten oder schnelle Reaktionen? Enthält das Programm Zufall? Enthält das Programm versteckte Informationen, die selektiv aufgedeckt werden?
    \item \emph{Fantasie}: Enthält das Programm eine emotional ansprechende Fantasie? Hängt die Fantasie instinktiv mit der in der Aktivität erlernten Fähigkeit zusammen? Ist die Fantasie eine nützliche Metapher?
    \item \emph{Neugierde}: Gibt es audio- und visuelle Effekte, um die Neugier der Sinne zu stimulieren? Gibt es Elemente, die die kognitive Neugier wie Überraschungen oder konstruktives Feedback stimulieren?
\end{itemize}

Diese Anforderungen sollten bei der Aufstellung der Anforderungen für das im Rahmen dieser Arbeit entwickelte Programm berücksichtigt werden, auch wenn sich aufgrund der für diese Art von Lernspiel notwendigen, im nächsten Abschnitt beschriebenen Mechaniken, nicht alle aufgestellten Anforderungen erfüllen lassen werden.

%************************************************
% Minisprachen
%************************************************
\section{Minisprachen}
\label{sec:basics:mini-languages}

Minisprachen sind Programmiersprachen, die speziell auf die Anforderungen von Programmieranfängern zugeschnitten sind, indem sie einen reduzierten Sprachumfang bieten, der speziell auf die Lösung einer bestimmten Klasse von Problemstellungen zugeschnitten ist und dabei die Grundprinzipien des Programmierens hervorhebt bzw. deren Erlernen fördert.

Warum das Erlernen von Programmierfähigkeiten mithilfe von Minisprachen im Gegensatz zu den breit genutzten Universalsprachen (wie z.~B. Java oder C) sinnvoll ist, erklären Brusilovsky et al., indem sie drei Probleme von Universalsprachen für die Anwendung zum Lernen nennen, die Minisprachen zu beheben versuchen \cite[67]{brusilovsky1997}:

\begin{itemize}
    \item Universalsprachen sind zu groß und zu idiosynkratisch. Die konzeptionelle Basis der Sprache bildet zusammen mit den Hauptprinzipien der Programmierung eine große Menge an Material. Anstatt die Grundprinzipien hervorzuheben, rufen die Sprachen nebensächlich Begriffe auf, die die Feinheiten der jeweiligen Sprache und deren Umsetzung, nicht aber die Hauptprinzipien der Programmierung widerspiegeln.
    \item Universalsprachen fördern nicht das Verständnis ihrer grundlegenden Aktionen und Kontrollstrukturen. Die Sprachen sind nicht visuell und ihre Grundfunktionen werden hinter einer undurchsichtigen Barriere ausgeführt. Wenn der Prozess der Programmausführung verborgen ist, entwickelt der Student ein Input-Output-orientiertes Verständnis. Auf diese Weise behindert das Fehlen von visuellem Feedback die Beherrschung der Sprachsemantik.
    \item Da sich Universalsprachen an der Zahlen- und Symbolverarbeitung orientieren, sind die ersten möglichen Aufgaben, die beim Unterrichten der Sprache umgesetzt werden können, weit von den Alltagserfahrungen der Schüler entfernt. Die Entwicklung von Anwendungen, die sowohl informativ als auch interessant sind, erfordert das Erlernen einer beträchtlichen Untermenge der Sprache und das Schreiben recht großer Programme.
\end{itemize}

Brusilovsky et al. heben in Ihrem Artikel \textit{Mini-languages: a way to learn programming principles} einige sich daraus ergebende Eigenschaften von Minisprachen als besonders wichtig hervor \cite[73-74]{brusilovsky1997}:

\begin{itemize}
    \item Sowohl Syntax, als auch Semantik der Sprache sollten \emph{einfach} sein.
    \item Die Operationen der Minisprache sollten \emph{sichtbar} sein. Die meisten Operationen, die der Akteur ausführt, sollten sichtbare Änderungen in der auf dem Bildschirm dargestellten Mikrowelt vornehmen.
    \item Die Minisprache sollte für die vorgesehene Kategorie von Studenten \emph{attraktiv} und aussagekräftig sein.
    \item Die Sprache sollte \emph{dialogorientiert} sein. Das bedeutet, dass die Sprache Befehl für Befehl in einem Navigationsmodus ausgeführt werden kann (Einzelbefehlsausführung) und als Ganzes in einem Programmiermodus (komplexe Programmausführung).
    \item Die Sprache sollte \emph{modular} sein. Sie sollte einen Mechanismus zum Erstellen abstrakter Anweisungen (Prozeduren) enthalten. Alle Verfahren sollten unabhängige Einheiten sein. Eine solche Prozedur kann als neuer Befehl des Akteurs betrachtet werden, der sowohl im Navigations- als auch im Programmiermodus verwendet werden kann.
\end{itemize}

Eine Mikrowelt ist dabei eine einfache Darstellung der von für die Minisprache benötigten Komponenten. In ihr findet keine realistische Simulation statt, sondern sie beschränkt sich auf einen leicht zu überblickenden Kern, der mit Brettspielen vergleichbaren Regeln folgt. Diese Regeln sind typischerweise mit der Minisprache abgestimmt.

Aus der Liste dieser Anforderungen ergeben sich direkte Anforderungen an die für diese Arbeit entwickelte Minisprache (siehe \ref{sec:requirements:program}). Sie sind auch -- soweit das beurteilt werden kann -- in allen im nächsten Kapitel genannten vergleichbaren Arbeiten berücksichtigt.
