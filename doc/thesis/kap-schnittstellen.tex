\section{Schnittstellen und Formate}
\info[inline]{Dieser Abschnitt richtet sich an Software-Entwickler, die Erweiterungen an esqulino vornehmen möchten.}

Dieses Kapitel beschreibt das Projektformat sowie die Kommunikation zwischen Server und Client. Da der Funktionsumfang des serverseitigen Teils vergleichsweise überschaubar gehalten werden kann, sollte sich nur auf Basis dieses Kapitels ein alternativer esqulino-Server implementieren lassen.

\subsection{Umgang mit Referenzen}

Prinzipiell soll jede nenutzerdefinierte Resource, also Abfrage oder Seite, beliebig umbenannt werden können, ohne das dadurch eine Re-organisation aller Verweise nötig wird. Deswegen werden solche Referenzen immer durch eine künstliche ID, technisch exakt eine UUID, hergestellt. Durch diesen Verzicht auf Benutzerdefinierte Attribute in den Verweisen können beliebige Umbenennungen oder Verschiebung der Hierarchieebenen mit wenig Aufwand unterstützt werden.

\subsection{Verwaltung von Projekten}

Um den Betrieb für Schüler und Lehrer zu vereinfachen, wird eine Instanz des Servers also in der Lage sein mehrere Schülerprojekte gleichzeitig bereitzustellen. Dafür ist es aber zunächst einmal notwendig zu definieren, wie ein solches Projekt überhaupt strukturiert ist. Grundsätzlich ist ein Projekt eine Sammlung von Dateien in einer festgelegten Ordnerstruktur.

\begin{dirstruct}
  \dirtree{%
    .1 empty-project/.
    .2 queries/.
    .2 pages/.
    .2 config.yaml.
    .2 db.sqlite.
  }
  \caption{Leeres Projekt}
\end{dirstruct}



%%% Local Variables:
%%% mode: latex
%%% TeX-master: "thesis"
%%% End:
