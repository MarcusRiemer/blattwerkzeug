\section{Projektbeispiele}
\label{sec:project-examples}

Dieses Kapitel umfasst einige Ideen für Projekte, die sich gut mit \textbf{\idename} umsetzen lassen. Diese Auflistung ist natürlich nicht vollständig, sie umfasst vielmehr jene Projekte die gewissermaßen als ``Proof-of-Concept'' im Rahmen der Entwicklung entstanden sind.

\missing[inline]{Infobox zu jedem Beispiel}
\missing[inline]{Datenbankschema zu jedem Beispiel}

\subsection{Interaktive Geschichten}

Im englischsprachigen Raum hat sich für diese Art von Erzählung der Terminus ``Choose Your Own Adventure'' durchgesetzt. In Deutschland exisitiert kein feststehender Begriff, stattdessen werden häufig exemplarische Buchreihen wie ``Insel der tausend Gefahren'' stellvertretend für das Genre herangezogen. Für all jene, die auch mit diesen Begriffen nichts anfangen können, illustriert der folgende Abschnitt, wie eine solche Geschichte funktioniert:

\missing[inline]{Beispielgeschichte}

\subsection{Historische Personen und Ereignisse}

Welche historischen Ereignisse hat eigentlich Walt Disney erlebt, als er zwischen 20 und 30 Jahre alt war? Und welche historischen Persönlichkeiten waren am Leben, als der Buchdruck erfunden worden ist?

Dank Quellen wie WikiData ist es mittlerweile vergleichsweise einfach, große Datenmengen automatisiert zu extrahieren. Zu diesem Projekt gehört daher auch ein kleines Skript, um die Datenbank mit Werten zu füllen.

\subsection{Sehr einfacher Blog mit Kommentaren}

%%% Local Variables:
%%% mode: latex
%%% TeX-master: "thesis"
%%% End:
