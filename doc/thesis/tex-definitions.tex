\newcommand{\doctitle}{esqulino}
\newcommand{\docsubtitle}{Eine datenzentrierte Entwicklungsumgebung für Schüler}
\newcommand{\docauthors}{Marcus Riemer}
\newcommand{\docpdfauthors}{Marcus Riemer}

\newcommand{\idename}{\textcolor{red}{e}\textcolor{orange}{s}\textcolor{GreenYellow}{q}\textcolor{green}{u}\textcolor{LimeGreen}{l}\textcolor{blue}{i}\textcolor{RoyalPurple}{n}\textcolor{violet}{o}}


\newcommand{\person}[1]{\textsc{#1}}

\newcommand*\circled[1]{\tikz[baseline=(char.base)]{
            \node[shape=circle,draw,inner sep=2pt] (char) {#1};}}

\usepackage{color}
\definecolor{lightgray}{rgb}{.9,.9,.9}
\definecolor{darkgray}{rgb}{.4,.4,.4}
\definecolor{purple}{rgb}{0.65, 0.12, 0.82}

\lstdefinelanguage{JavaScript}{
  keywords={typeof, new, true, false, catch, function, return, null, catch, switch, var, if, in, while, do, else, case, break},
  keywordstyle=\color{blue}\bfseries,
  ndkeywords={class, export, boolean, throw, implements, import, this, constructor, public, private},
  ndkeywordstyle=\color{darkgray}\bfseries,
  identifierstyle=\color{black},
  sensitive=false,
  comment=[l]{//},
  morecomment=[s]{/*}{*/},
  commentstyle=\color{purple}\ttfamily,
  stringstyle=\color{red}\ttfamily,
  morestring=[b]',
  morestring=[b]"
}

\lstset{
   language=SQL,
   backgroundcolor=\color{lightgray},
   extendedchars=true,
   basicstyle=\footnotesize\ttfamily,
   showstringspaces=false,
   showspaces=false,
   numberstyle=\footnotesize,
   numbersep=9pt,
   tabsize=2,
   breaklines=true,
   showtabs=false,
   captionpos=b
}

\lstset{escapeinside={(*@}{@*)}}

\newcommandx{\unsure}[2][1=]{\todo[linecolor=red,backgroundcolor=red!25,bordercolor=red,#1]{\textbf{Unsure}: #2}}
\newcommandx{\change}[2][1=]{\todo[linecolor=blue,backgroundcolor=blue!25,bordercolor=blue,#1]{\textbf{Change}: #2}}
\newcommandx{\info}[2][1=]{\todo[linecolor=OliveGreen,backgroundcolor=OliveGreen!25,bordercolor=OliveGreen,#1]{\textbf{Info}: #2}}
\newcommandx{\missing}[2][1=]{\todo[linecolor=Plum,backgroundcolor=Plum!25,bordercolor=Plum,#1]{\textbf{Missing}: #2}}

\newcommand{\warning}[2][Achtung]{
  \begin{framed}
    \textbf{#1}: #2
  \end{framed}
}

% Using \DeclareFloatingEnvironment leads to a strange extra dot in the
% caption numbering, see http://tex.stackexchange.com/questions/330638/getting-rid-of-an-extra-dot-in-the-numbering-of-my-new-float-environment

\newfloat{diagram}{thp}{lop}
\floatname{diagram}{Diagramm}


%% Citing stuff
\usepackage[backend=biber,defernumbers=true]{biblatex}
\addbibresource[datatype=bibtex]{library.bib}

%%% Local Variables:
%%% mode: latex
%%% TeX-master: "thesis"
%%% End:
