\begin{tikzpicture}
  \tikzset{square matrix/.style={
      column sep=-\pgflinewidth, row sep=-\pgflinewidth,
      nodes={
        minimum height=#1,
        anchor=center,
        text width=#1,
        align=center,
        inner sep=6pt
      },
    },
    square matrix/.default=3.50cm
  }


  \matrix[square matrix] (my matrix) at (0,0)
  {
    \node (single)   {Unmittelbare Ausgabe, z.B. in einem Text}; &
    \node (passive)  {Einfaches Eingabelement, z.B. eine Textbox}; \\
    \node (multiple) {Wiederholte Ausgabe, z.B. eine Tabelle}; &
    \node (input)    {Eingabelement mit Mehrfachauswahl}; \\
  };
  \draw [thick,-] (my matrix.east)  |- (my matrix.west);
  \draw [thick,-] (my matrix.south) |- (my matrix.north);

  \node [left=of single, anchor=north, rotate=90] {\textsc{Eine Zeile}};
  \node [left=of multiple, anchor=north, rotate=90] {\textsc{Beliebig}};
  \node [above=of single, anchor=north] {\textsc{Keine Eingabe}};
  \node [above=of passive, anchor=north] {\textsc{Eingabe}};
\end{tikzpicture}



%%% Local Variables:
%%% mode: latex
%%% TeX-master: "../thesis"
%%% End:
