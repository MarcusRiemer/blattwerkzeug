\section{Umsetzungsanalyse}
\label{sec:implementation-analysis}

\info[inline]{Dieser Abschnitt richtet sich an Software-Ingenieure, die Erweiterungen am bestehenden Code von esqulino vornehmen möchten.}

\subsection{Verwendete Sprachen und Bibliotheken}

Der softwaretechnische Unterbau der Entwicklungsumgebung wurde schon vor dem eigentlichen Beginn der Thesis in Gesprächen mit Dr. Huch und Dr. Hoffmann festgelegt.

\begin{description}
\item[Server: Ruby mit Sinatra] \hfill\\
  Die Aufgaben des Servers sollen sich konzeptionell möglichst auf die Auslieferung und Speicherung von Daten beschränken. Die Interaktion findet dabei primär über eine REST-artige JSON Schnittstelle statt.
\item[Client: Typescript mit Angular 2] \hfill\\
  Aufgrund des hohen Grades an Interaktivität bietet sich eine rein clientseitige Visualisierung an, die weitestgehend auf Roundtrips zum Server verzichtet. Zu Beginn der Arbeit befand sich Angular 2 noch in der Betaphase.
\end{description}

\subsection{Hinweise zum Client}

Der verwendete Typescript Compiler hat zum Zeitpunkt der Anfertigung dieser Arbeit einen bekannten Bug in der Codegenerierung \cite{ts-compiler-class-order-bug} um den wiederholt herumgearbeitet werden musste. Konkret äussert sich dieser Fehler, wenn die Definition der Oberklasse einer sich davon ableitenden Klasse erst im Nachhinein erfolgt (Listing \ref{lst:ts:class-order-bug}). In diesem Fall kommt es zu keiner Warnung durch den Compiler, sondern zu einem Laufzeitfehler im kompilierten Javascript-Code.

\lstinputlisting[language=JavaScript,caption=Falsche Reihenfolge der Klassendefinition, label=lst:ts:class-order-bug]{snippets/class-inheritance-order-bug.ts}

%%% Local Variables:
%%% mode: latex
%%% TeX-master: "thesis"
%%% End:
