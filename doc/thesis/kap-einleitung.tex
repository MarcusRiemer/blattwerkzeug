\section{Einleitung}

In Deutschland scheint man Informatik als eine Hochschulangelegenheit zu betrachten. Zwar ist Informatikunterricht in der Schule vorgesehen, der Begriff ``Informatik'' wird dabei aber oftmals sehr weit interpretiert. Natürlich wird der Umgang mit Textverarbeitung, Tabellenkalkulation und Präsentationsmitteln heutzutage an fast jeder Schule zumindest thematisiert, Inhalte mit tatsächlichem Bezug zur Informatik sind das aber nur in Ausnahmefällen. Schaut man sich die für Informatikinhalte zur Verfügung stehenden Programme an (natürlich sollte Informatikunterricht auch am Rechner stattfinden!), verwundert diese stiefmütterliche Behandlung nur wenig: Die Menge an verfügbarer Software ist ausgesprochen überschaubar. Und natürlich ist es keinem Informatiklehrer zuzumuten, diese Lücke immer mit eigens geschriebener Software zu füllen!

Diese Arbeit ist ein Versuch, eine Lehrsoftware für zwei wichtige Teilgebiete der Informatik anzubieten: Datenmodellierung und Anwendungsentwicklung. Ganz konkret geht es um die Konzeption und Implementierung einer einsteigerfreundlichen Entwicklungsumgebung für den anwendungsorientierten Umgang mit SQL-Datenbanken. Zielgruppen dieser Software sind Schülerinnen und Schüler ab der Mittelstufe sowie natürlich deren Lehrkräfte. Die Anwender der Software sollen in die Lage versetzt werden, zu einem gegebenen Datenmodell sowohl inhaltliche Fragen mit Bezug zu einem konkreten Datenbestand zu beantworten als auch neue Daten in das Modell einzupflegen. Darüber hinaus soll es Ihnen möglich sein, die Datenbank über eine eigens von den Nutzern entwickelte Oberfläche auch einer begrenzten Öffentlichkeit zugänglich zu machen.

\subsection{Eine spezielle Entwicklungsumgebung}

Die wesentliche Anforderung an die im Rahmen dieser Arbeit zu erstellende Software ergibt sich also schon aus dem Titel der Arbeit: Es geht vorrangig um die Vermittlung von praktischen Kenntnissen zur Abfrage- und Manipulation von komplexen Datenbeständen in Anlehnung an die Projektideen der Lehrpläne \cite{lehrplan-inf-sek-1} bzw. Fachanforderungen \cite{lehrplan-inf-sek-2} für Informatik des Landes Schleswig-Holstein\footnote{Die exakte Verortung oder auch die Konzeption von konkreten Schulstunden ist hingegen nicht Teil dieser Arbeit.}. Auch wenn sich aktuell eine zunehmende Pluralität von Paradigmen zur Datenspeicherung abzeichnet, die das relationale Modell in Frage stellen oder ergänzen, behandelt diese Arbeit explizit die Vermittlung von SQL-Kenntnissen. 

Da der Inhalt dieser Arbeit die Konzeption und Umsetzung einer Software ist, deren Zweck wiederum die Entwicklung anderer Software ist, bedarf es zunächst eines gemeinsamen Verständnisses für die Bezeichnungen der beteiligten Systeme.

\begin{description}
\item[(Schüler-)Entwicklungsumgebung] \hfill\\ 
  Bezeichnet die von den Lernenden zu nutzende Software, die im Rahmen dieser Arbeit erstellt wurde. Sofern Verwechslungen mit anderen Entwicklungsumgebungen (engl. ``integrated development environment'', IDE) auftreten könnten, wird explizit das Präfix ``Schüler'' genutzt. Letztere Unterscheidung wird insbesondere bei Vergleichen relevant sein.
\item[(Schüler-)Projekt] \hfill\\
  Bezeichnet die von den Lernenden, unter Nutzung der Schülerentwicklungsumgebung, erstellte Software. Dazu gehört unter anderem das Datenbankschema, die verschiedenen Abfragen und die gestaltete Benutzeroberfläche.
\item[Endanwender] \hfill\\
  Bezeichnet Personen, die mit der ``normalen'' Benutzeroberfläche des Schülerprojekts interagieren, nicht jedoch mit dem Editor selbst.
\end{description}
%%% Local Variables:
%%% mode: latex
%%% TeX-master: "thesis"
%%% End:
