\section{Einleitung}

In Deutschland scheint man Informatik als eine Hochschulangelegenheit zu betrachten. Anders ist es für mich kaum zu erklären, dass der Informatikunterricht an Schulen zwar durchaus vorgesehen ist, der Begriff ``Informatik'' wird dabei aber oftmals sehr weit interpretiert. Der Umgang mit Textverarbeitung, Tabellenkalkulation und Präsentationsmitteln wird heutzutage zwar an fast jeder Schule zumindest thematisiert, Inhalte mit tatsächlichem Bezug zur Informatik wie sie in einer späteren Ausbildung oder im Studium relevant relevant wären, sind dabei aber in der Minderheit. Schaut man sich die für Informatikinhalte zur Verfügung stehenden Lehr- und Lernprogramme an (natürlich sollte Informatikunterricht auch am Rechner stattfinden!), verwundert diese stiefmütterliche Behandlung nur wenig: Die Menge an verfügbarer und gepflegter Software ist ausgesprochen überschaubar. Und natürlich ist es keinem Informatiklehrer zuzumuten, diese Lücke mit eigens geschriebener Software zu füllen.

Diese Arbeit ist ein Versuch, eine Lehrsoftware für zwei wichtige Teilgebiete der Informatik anzubieten: Datenmodellierung und (Web-)Oberflächenentwicklung. Ganz konkret geht es um die Konzeption und Implementierung einer einsteigerfreundlichen Entwicklungsumgebung für den anwendungsorientierten Umgang mit SQL-Datenbanken. Zielgruppen dieser Software sind Schülerinnen und Schüler an weiterführenden Schulen sowie natürlich deren Lehrkräfte. Die Anwender der Software sollen in die Lage versetzt werden, zu einem gegebenen Datenmodell sowohl inhaltliche Fragen mit Bezug zu einem konkreten Datenbestand zu beantworten als auch neue Daten in das Modell einzupflegen. Darüber hinaus soll es ihnen möglich sein, die Datenbank über eine eigens entwickelte Oberfläche einer begrenzten Öffentlichkeit zugänglich zu machen.

\subsection{Eine spezielle Entwicklungsumgebung}

Der wesentliche Anspruch an die im Rahmen dieser Arbeit zu erstellende Software ergibt sich also schon aus dem Titel der Arbeit: Es geht vorrangig um die Vermittlung von praktischen Kenntnissen zur Abfrage und Manipulation von komplexen Datenbeständen in Anlehnung an die Projektideen der Lehrpläne \cite{lehrplan-inf-sek-1} bzw. Fachanforderungen \cite{lehrplan-inf-sek-2} für Informatik des Landes Schleswig-Holstein\footnote{Die exakte Verortung im Curriculum oder auch die Konzeption von konkreten Schulstunden ist hingegen nicht Teil dieser Arbeit.}. Auch wenn sich aktuell eine zunehmende Pluralität von Paradigmen zur Datenspeicherung abzeichnet, welche das relationale Modell ergänzen oder in Frage stellen, behandelt diese Arbeit explizit die Vermittlung von SQL-Kenntnissen. 

Da der Zweck dieser Arbeit die Konzeption und Umsetzung einer Software ist, deren Zweck wiederum die Entwicklung anderer Software ist, bedarf es zunächst eines einheitlichen Verständnisses für die Bezeichnungen der beteiligten Personen und Entitäten:

\begin{description}
\item[(Schüler-)Entwicklungsumgebung, esqulino] \hfill\\ 
  Bezeichnet die von den Lernenden zu nutzende Software, die im Rahmen dieser Arbeit erstellt wird. Sofern Verwechslungen mit anderen Entwicklungsumgebungen (engl. ``integrated development environment'', IDE) auftreten könnten, wird explizit das Präfix ``Schüler'' genutzt. Diese Unterscheidung wird insbesondere bei Vergleichen mit gängiger Entwicklungssoftware Art relevant sein.
\item[(Schüler-)Projekt] \hfill\\
  Bezeichnet die von den Lernenden unter Nutzung der Schülerentwicklungsumgebung erstellte Software. Teil eines solchen Projektes sind unter anderem das Datenbankschema, die verschiedenen Abfragen und die gestaltete Benutzeroberfläche.
\item[Entwickler] \hfill\\
  Bezeichnet Personen, die mit der ``entwickelnden'' Benutzeroberfläche eines Schülerprojekts interagieren.
\item[Endanwender] \hfill\\
  Bezeichnet Personen, die mit der von den Entwicklern erstellten ``normalen'' Benutzeroberfläche eines Schülerprojekts interagieren, nicht jedoch mit der Schülerentwicklungsumgebung selbst.
\end{description}
%%% Local Variables:
%%% mode: latex
%%% TeX-master: "thesis"
%%% End:
