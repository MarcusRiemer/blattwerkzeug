\section{Vergleichbare Arbeiten}

Andere Entwicklungsumgebungen für Datenbanken und auch Generatoren für Abfragemasken gibt es zuhauf. Dieses Kapitel stellt einige der vefügbaren Programme vor, insbesondere im Hinblick auf für diese Arbeit formulierten Prinzipien und unter Betrachtung der avisierten Zielgruppe.

\unsure[inline]{Momentan eher eine sehr lose Sammlung denn eine tatsächliche Recherche.}

\subsection{Software: Scratch}

\info[inline]{Paradebeispiel für eine schülerorientierte Entwicklungsumgebung, insbesondere auch eine Betrachtung der visuellen Programmierung.}

\subsection{Software: Visual Studio Lightswitch}

\info[inline]{Generator für Datengetriebene Geschäftsanwendung mit überschaubarer Applikationslogik.}

\subsection{Buch: Die Macht der Abstraktion}

\info[inline]{Buch zum Einstieg in die Programmierung mit einem interessanten, sehr Datentypgetriebenen Ansatz.}

\subsection{Software: Microsoft Access}

\info[inline]{Eher eine Datenbank als eine sinnvolle Eingabemaske für unversierte Benutzer.}

\subsection{Software: MySQL Workbench}

\info[inline]{Wirklich ein Datenbanktool, keinerlei Benutzerschnittstelle für ``normale'' Benutzer.}

%%% Local Variables:
%%% mode: latex
%%% TeX-master: "thesis"
%%% End:
